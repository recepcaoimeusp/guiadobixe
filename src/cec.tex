\begin{secao}{CEC - Centro de Ensino de Computação}

\begin{subsecao}{O que é?}

O Centro de Ensino de Computação (CEC) foi criado em 1993, pelo
Departamento de Ciência de Computação do IME, com o objetivo de
oferecer apoio às aulas dos cursos de graduação do Instituto e para os
treinamentos/cursos oferecidos à comunidade USP e não USP, através de
seus laboratórios com recursos de informática.
Está localizado no Bloco B do IME, ao lado da Seção de Alunos. O horário
de funcionamento do CEC é de 2a a 6a das 8h00 às 22h00 e nos recessos
escolares o horário de abertura é alterado para 9h00.

\end{subsecao}

\begin{subsecao}{Estrutura}

O CEC possui 137 computadores, distribuídos em 04 (quatro) salas/laboratórios,
sendo 02 (duas) delas com capacidade para até 60 pessoas e com prioridade de uso
para as aulas de graduação do IME e para os cursos/treinamentos. Os outros
laboratórios são utilizados por alunes de graduação dos cursos do IME para a
realização dos trabalhos e pesquisas acadêmicas. Em todos os computadores o Sistema
Operacional é o Linux Debian e possuem os aplicativos solicitados pelos docentes
do Instituto.

\end{subsecao}

\begin{subsecao}{Como funciona?}

Possui rede cabeada e para utilizá-la é necessário ter uma senha de
acesso (conta de usuário). Você precisa ser alune matriculado no IME para
solicitar a senha de uso, pois o CEC não é uma sala pró-aluno (para ver como as 
salas pró-aluno funcionam acesse \url{https://www5.usp.br/keywords-s/salas-pro-aluno/}). 
Para solicitar a senha de acesso, envie e-mail para \url{cec-senha@ime.usp.br},
com o assunto "CEC senha". Informe seu nome completo e número USP. Utilize
seu e-mail institucional (@usp.br - se esqueceu entre em
\url{https://id.usp.br/}).

\end{subsecao}

\textbf{Obs.:} bixes, ao frequentarem o CEC, fiquem atentes ao ar-condicionado. Se
estiver ligado, deem preferência a usarem calças, blusas, jaquetas e meias de
lã. Cobertores são opcionais. Se não, boa sorte ou \textit{hasta la vista}!

\end{secao}
