\begin{subsecao}{Estatística}

Muito bem, bixes, vocês conseguiram passar em Estatística! Depois de enfrentar 
o vestibular e aquela dúvida existencial de ``o que exatamente faz ume 
estatístique?'', vocês finalmente chegaram não apenas ao IME, mas ao curso mais 
legal! Hora de descansar, respirar fundo e curtir a vida universitária.

Vocês logo vão perceber que Estatística não é sobre fazer continha em planilha, 
muito menos sobre apertar botões aleatórios em softwares mágicos que cospem 
gráficos coloridos. Aqui, vocês vão aprender a pensar de verdade ao lidar com 
dados e isso envolve uma quantidade respeitável de Matemática, Computação, 
Probabilidade, Estatística e, claro, algumas crises ao tentar entender como os 
resultados apresentados são possíveis.

Ao longo do curso, vocês vão descobrir que Estatística é a ciência (ou a arte) 
de extrair sentido do caos: modelar incertezas, lidar com variabilidade e tomar 
decisões quando a única certeza é que nada é 100\% certo. Parece filosófico? É. 
Mas também é extremamente prático e, por isso mesmo, exige uma base teórica 
sólida. Infelizmente, nem tudo nessa vida é 50\%.

Vamos começar com um pouquinho da história do curso: o primeiro bacharelado em 
Estatística (BE) no Brasil foi criado em 1953, na Escola Nacional de Ciências 
Estatísticas (ENCE), vinculada ao IBGE, o que trouxe um grande destaque à área 
no país. Já no nosso IME-USP, o departamento de Estatística foi fundado alguns 
anos mais tarde, em 1970, em virtude das reformas universitárias que alteraram 
toda a estrutura da USP.

Em $1972$, o grandioso bacharelado em Estatística (BE) foi criado tendo como 
base o currículo da ENCE. O ingresso no curso era feito junto dos alunes do 
bacharelado em matemática e, após um ano, poderiam optar pela Estatística. 
Curiosamente, nessa época, engenharia e matemática (e Estatística) faziam parte 
da mesma carreira na FUVEST, o que afetava drasticamente o perfil dos alunes do 
curso. Posteriormente, separaram as carreiras e, por fim, Estatística tornou-se 
um curso separado do bacharelado em matemática.

Docentes e discentes do curso têm por objetivo modificar a grade curricular a 
cada 10 anos para acompanhar a rápida evolução do mercado, das técnicas 
computacionais e da teoria. A última reforma aconteceu em 2022, logo após o fim 
da pandemia. A nova grade acabou tornando o primeiro e o segundo ano do 
bacharelado um pouco mais pesados, mas aliviou significativamente o terceiro 
ano, que era tido como quase impossível de ser concluído no tempo ideal. Como 
vocês são a quinta turma a vivenciar essa estrutura, podem apenas aproveitar e 
tentar seguir a grade sugerida.

\vspace{10pt}
\textbf{\Large Estrutura do Curso}
\vspace{5pt}

O primeiro ano do curso é caracterizado por disciplinas de caráter 
fundamentalmente teóricas.

\begin{itemize}
    \item \textbf{Cálculo I \& II}: a mais famosa das disciplinas dos cursos de 
    exatas obviamente estará presente no bacharelado em Estatística (BE). Nosso 
    cálculo é exclusivo para a estatística, pois a ementa é voltada às 
    necessidades reais de um estatístico;
    \item \textbf{Vetores e Geometria \& Álgebra Linear}: são uma parte 
    fundamental para lidar com matrizes, que nada mais são do que conjuntos 
    enormes de dados! Preste muita atenção, pois saber alguns bons truques de 
    Algelin poderão te ajudar muito nas próximas disciplinas;
    \item \textbf{Introdução à Computação \& Princípios de Desenvolvimento de 
    Algoritmos}: duas disciplinas que te fornecerão uma base excelente para 
    programar. Os cursos geralmente são oferecidos em \textit{Python} e com 
    algumas dinâmicas bem inusitadas para avaliar os alunes;
    \item \textbf{Probabilidade I \& II}: finalmente as primeiras matérias 
    únicas do BE! Lembra de tudo que você aprendeu no ensino médio? Não descarte 
    ainda, mas haverá um mundo completamente novo. As primeiras aulas podem ser 
    puxadas, mas eu garanto que você ficará surpreso em como a disciplina mudará 
    em algumas poucas semanas;
    \item \textbf{Análise Exploratória de Dados \& Introdução à Inferência 
    Estatística}: aqui vocês terão um primeiro gostinho do que é Estatística na 
    prática, mas ainda muito voltada à teoria. Fiquem tranquilos, pois 
    absolutamente tudo que virem aqui será aprofundado em disciplinas futuras.
\end{itemize}

Do segundo ano em diante, a maioria absoluta das disciplinas será 
significativamente mais aplicadas ou notoriamente menos teóricas, o que não 
quer dizer que não haverá muitos mais teoremas para decorar. Dizem que ume 
estudante do BE só pode dizer se realmente gosta de estatística ao final do 
segundo ano, pois é quando finalmente tem contato com as três grandes áreas da 
Estatística.

Naturalmente, você deve estar se perguntando "O que são as três grandes áreas 
da Estatística?". Primeiramente, você deve entender que estamos adentrando um 
campo muito delicado, escolher uma das áreas é como escolher seu time de 
futebol: nem sempre é uma escolha racional, mas muitos defenderão com unhas e 
dentes. Aqui vão as três grandes áreas:

\begin{itemize}
    \item \textbf{Bayesiana}: caracterizada por enxergar a probabilidade como 
    uma medida de crença. Entende que o fazer científico carrega a 
    subjetividade do pesquisador e que devemos atualizar nossas conclusões à 
    medida que novos dados são obtidos (o famoso "conhecimento a priori");
    \item \textbf{Frequentista}: enxerga a probabilidade como a frequência de 
    um evento após infinitas repetições. Diferente dos bayesianos, tratam os 
    parâmetros como fixos, um número que só existe no mundo das ideias, e 
    buscam uma objetividade rigorosa, baseando suas conclusões apenas nos dados 
    observados no experimento atual;
    \item $\mathbf{Probabilista} \ \star$: área favorita deste que vos escreve, 
    é a mais próxima da matemática pura. Para os probabilistas, a probabilidade 
    é um objeto matemático e eles não se preocupam tanto com interpretações 
    filosóficas. Seu foco está na estrutura lógica do aleatório, o que muitas 
    vezes os direciona mais à pesquisa acadêmica do que à análise direta de 
    dados do cotidiano.
\end{itemize}

Mesmo após esse resumo ainda não sabe qual \sout{time} área escolher? Não se 
preocupe, nos dois primeiros anos você conhecerá um pouquinho de cada e, se 
quiser, poderá finalmente escolher a que mais se adaptar.

Se você está cansado de tomar decisões, tenho boas e más notícias: o BE oferece 
um certificado extra, além do diploma, que atesta sua especialização em até 
duas áreas a sua escolha como Ciência de Dados, Economia, Atuária ou 
Fundamentos de Estatística. Para garantir esse reconhecimento, você deve cursar 
pelo menos quatro disciplinas relacionadas ao tema escolhido (a lista de 
disciplinas está no catálogo de graduação). A vantagem é que você não precisa de 
tempo extra, já que essas disciplinas podem ser aproveitadas dentro da carga 
horária de optativas eletivas e livres que você já precisa cumprir. Aqui estão 
mais alguns detalhes das áreas:

\begin{itemize}
    \item \textbf{Ciência de Dados}: Foca em habilidades de computação de alto 
    desempenho e manipulação de grandes volumes de dados (Big Data), 
    aprofundando conhecimentos em processamento, otimização e paralelização de 
    algoritmos.

    \item \textbf{Fundamentos de Estatística e Probabilidade}: Destinada a quem 
    pretende seguir carreira acadêmica ou fazer pós-graduação, oferecendo a base 
    teórica avançada necessária para a pesquisa científica nessas áreas.

    \item \textbf{Economia}: Introduz os pilares da Micro e Macroeconomia, além 
    da Econometria, preparando o aluno para aplicar a estatística no mercado 
    financeiro, análise de riscos e formulação de políticas públicas.

    \item \textbf{Atuária}: Fornece uma base em seguros e finanças, unindo o 
    rigor quantitativo da estatística com a gestão de riscos e o planejamento 
    financeiro de longo prazo.
\end{itemize}

Agora que você entendeu o início e o meio do curso, como é o último ano? O 
"chefão final" é a disciplina de Estatística Aplicada I \& II, a única 
obrigatória do último ano (geralmente às terças e quintas à tarde). Nela, você 
terá a experiência mais próxima possível do mercado de trabalho ou da pesquisa 
real: a consultoria estatística. Tal disciplina está vinculada ao Centro de 
Estatística Aplicada (CEA), que é um Centro de Consultoria, por meio do qual 
docentes do Departamento de Estatística prestam serviços de consultoria e 
assessoria em Estatística Aplicada. As coisas sempre acabam se misturando e 
apelidamos a disciplina de CEA.

O formato é direto ao ponto: docentes trazem clientes reais, que podem ser 
pesquisadores de outros departamentos, instituições públicas ou empresas, e cada 
grupo de alunes assume um projeto. Você precisará realizar entrevistas para 
entender o problema, planejar o estudo, analisar os dados e, por fim, apresentar 
os resultados e entregar um relatório formal (que ficará guardado na biblioteca 
do IME!). É o momento de enfrentar problemas e dados reais que os livros 
didáticos não mostram, mas fique tranquilo: você já terá todas as ferramentas 
necessárias para resolver os desafios que aparecerem.

\vspace{10pt}
\textbf{\Large Sobrevivendo ao Curso}
\vspace{5pt}

Infelizmente, o currículo da Estatística é conhecido por ser uma estrutura de 
"castelo de cartas": muitas matérias trancam várias outras lá na frente. Como as 
disciplinas são extremamente dependentes entre si, reprovar ou trancar uma
matéria de base no início do curso pode gerar um efeito cascata que atrasa sua 
formatura em um ou dois anos. Você precisa de uma base sólida em Cálculo e 
Probabilidade para sequer entender o que está acontecendo nas matérias 
específicas do meio do curso.

Antes de tomar qualquer decisão drástica sobre trancar uma disciplina, procure 
as pessoas veteranas: elas são o seu melhor recurso para entender o IME. Como 
alguns semestres são reconhecidamente mais pesados que outros, tentar carregar 
todas as obrigatórias de uma vez pode ser uma receita para o burnout. A 
estratégia ótima é identificar com quem já passou pelo aperto quais são as 
poucas disciplinas "soltas": aquelas que não trancam quase nada e que você 
pode deixar para o final do curso, permitindo que você foque nas matérias 
que realmente ditam o ritmo da sua progressão. Muitas vezes, uma disciplina 
que parece impossível agora é o único requisito para várias outras no semestre 
seguinte; nesses casos, o esforço para passar (mesmo que raspando) vale muito 
mais a pena do que um trancamento que atrasaria sua formatura em um ano. E por 
que não entregamos a lista dessas matérias mastigada aqui? Porque queremos que 
você perca a vergonha, interaja e conheça seus veteranos!

Tá com as matérias em dia e percebeu que tá precisando de dinheiro? Temos 
algumas boas opções para você: monitorias, bolsas de pesquisa e aulas 
particulares.

No IME, pouquíssimas pessoas batem na porta de docentes para pedir bolsas PUB 
(Programa Unificado de Bolsas), ainda menos no departamento de Estatística. Isso 
abre uma janela de oportunidade incrível para quem está no primeiro ano: basta 
achar um docente que você tenha afinidade, encontrar um tema que interesse a 
ambos e ter um pouco de 'cara de pau'. Uma dica de ouro é consultar a área de 
pesquisa do docente no Lattes antes de ir conversar. As inscrições costumam 
acontecer no meio do ano, então fique atento. Já outras bolsas, como Fapesp e 
CNPq, costumam ser mais restritas a veteranos por serem mais concorridas ou 
exigirem requisitos técnicos que você ainda não tem.

Ademais, há a possibilidade de ser monitor. Para bixes do primeiro ano, a única 
chance real geralmente é a monitoria de Probabilidade I, que é reoferecida no 
segundo semestre. Como o currículo da Estatística é muito mais profundo do que o 
de qualquer outro curso, a gente acaba vendo os tópicos com um rigor que os 
outros cursos não atingem de imediato.

A partir do segundo ano, o jogo vira e as opções explodem. Você pode ser 
monitor de qualquer matéria que já tenha cursado, além do famoso 'Grupão', 
uma disciplina de estatística básica oferecida para a galera das Ciências 
Sociais, Biologia, Oceanografia, Psicologia, Farmácia, Educação Física e alguns 
outros cursos. Outra opção são as disciplinas de estatística da FEA. Você vai 
perceber que essas matérias são grandes simplificações do que você estuda; 
afinal,quase todo mundo na USP tem estatística na grade, mas pouquíssimos 
realmente entendem o que estão fazendo. Você, por outro lado, faz parte dessa 
elite que domina a ciência por trás dos dados! Quer conselhos sobre docentes 
e disciplinas pra ser monitor sem pesar pro seu lado? Não preciso nem dizer que 
uma boa opção é consultar seus veteranos.

Por fim, se você quer bastante flexibilidade e uma grana imediata, as aulas 
particulares são um caminho. O seu conhecimento matemático e estatístico é 
muito valioso, até mesmo para estudantes de outras faculdades como Insper, FGV 
e PUC. Os alunes frequentemente travam em matérias de exatas, especialmente 
estudantes de cursos de humanas e você pode ajudá-los com isso. Saiba que o 
valor da hora-aula para esse público costuma ser muito recompensador.

\vspace{10pt}
\textbf{\Large Depois de Formado}
\vspace{5pt}

Parabéns, você sobreviveu! Mas o que te espera do lado de fora? A boa notícia é 
que você escolheu o curso certo na hora certa. A Estatística é, frequentemente, 
listada como uma das profissões que mais crescem no mundo, impulsionada pela 
onipresença dos dados em todos os segmentos, especialmente nos mais rentáveis.

Se você tem medo de ficar desempregado, pode respirar fundo: o mercado para 
estatístiques da USP é um dos mais fáceis de se adentrar. A demanda é tão alta 
que muitas vezes o emprego vem até você. A USP é palco constante de feiras de 
profissões e o IME abriga muitos eventos organizados pelas próprias empresas 
(como bancos, consultorias e gigantes da tecnologia), que vão até o instituto 
disputar os alunes.

A rede de contatos (o famoso \textit{networking}) é outro diferencial absurdo. 
No grupo da Estatística, alunes e ex-alunes divulgam vagas em diversos setores, 
especialmente no mercado financeiro e de seguros. E a melhor parte? Muitos 
ex-alunes, que hoje ocupam cargos de liderança em grandes empresas, priorizam 
indicar quem também veio do IME, pois reconhecem a qualidade e têm uma certa 
paixão pelo nosso, apesar de tudo, querido IME-USP. Além disso, fique de olho: 
existem inúmeras vagas e programas de carreira que são exclusivos para alunes da 
USP. As empresas sabem que ter esse nome no currículo é um excelente atestado de 
qualidade não apenas nos jantares em família.

Para quem pretende seguir o caminho acadêmico, o IME oferece programas de 
Mestrado, Doutorado e Doutorado Direto. Atualmente, o departamento incentiva o 
ingresso direto no doutorado para alunes da casa, simplificando o processo de 
seleção e, em muitos casos, oferecendo bonificações. A transição é facilitada 
pela própria natureza do bacharelado na USP: o currículo é tão abrangente e 
qualificado que muitos tópicos da graduação equivalem ao nível de pós-graduação 
de outras instituições.

É muito fácil encontrar ex-alunes em universidades estrangeiras, muitas vezes 
com bolsas de estudo integrais. Para quem deseja seguir a carreira docente e se 
apaixonou pela Cidade Universitária, o instituto demonstra uma forte tradição de 
valorizar seus egressos em concursos e contratações, o que torna a permanência 
no IME uma trajetória viável. Seja para consolidar uma carreira de pesquisador 
no exterior ou para tornar-se docente na própria instituição, você encontra 
uma base que sustenta ambas as opções. Por fim, lembre-se que uma carreira não 
exclui a outra: um profissional pós-graduado em estatística é extremamente 
concorrido em qualquer setor.

\vspace{10pt}
\textbf{\Large Estatística Além das Disciplinas}
\vspace{5pt}

A essa altura do campeonato, você já deve saber das inúmeras opções de 
atividades que o IME e a USP oferecem para você aproveitar. Mas e se você 
quiser atividades voltadas especificamente para a Estatística? O curso vai muito 
além da sala de aula, e participar dessas iniciativas é o que te diferenciará ao 
pleitear bolsas e fazer pedidos de financiamento para pesquisas.

\begin{itemize}
    \item \textbf{Palestras do Grupo de Probabilidade}: o Grupo de Probabilidade 
    é um dos braços de pesquisa mais fortes do IME. Eles realizam seminários 
    constantes sobre os temas mais atuais na área. Sendo honesto, você 
    provavelmente não entenderá a maior parte dos tópicos durante o primeiro ano, 
    mas o esforço vale a pena para se familiarizar com a linguagem e, 
    eventualmente, conhecer um bom orientador para uma futura Iniciação 
    Científica (IC). As reuniões costumam ocorrer no NUMEC, um prédio de acesso 
    bem restrito, mas que você poderá visitar se for assistir a uma palestra. 
    Os horários mudam a cada semestre, então a dica é chegar em alguém do grupo 
    e perguntar sobre as próximas datas ou dar a sorte de encontrar veteranes 
    que frequentem essas reuniões;
    \item \textbf{aMostra de Estatística}: a aMostra é o evento anual organizado 
    por estudantes do curso. A ideia é criar um espaço descontraído para 
    discutir o dia a dia da profissão através de palestras, mesas-redondas e 
    minicursos. Como alune, você pode atuar nos bastidores, fazendo parte da 
    comissão organizadora, ou apenas aproveitar o evento como ouvinte. Além de 
    conhecer empresas que vêm buscar talentos e distribuir brindes, você terá 
    acesso ao que muitos consideram o melhor coffee break da USP, as opções são 
    realmente divinas e é tudo totalmente gratuito;
    \item \textbf{Seminários do CEA}: lembra da disciplina de Estatística 
    Aplicada (CEA)? As apresentações dos veteranos são totalmente abertas e 
    você pode entrar na sala e ouvir sobre os projetos reais que chegam para 
    consultoria. Recomendamos fortemente que você assista a algumas. Mesmo que o 
    modelo estatístico pareça complexo no início, as primeiras palestras são 
    muito visuais, focadas em gráficos e tabelas que mostram os dados brutos dos 
    pesquisadores. Ver esses trabalhos inusitados na prática ajuda muito em 
    matérias da graduação, especialmente em Análise Exploratória de Dados. Os 
    horários estão disponíveis na secretaria do curso (no bloco A) ou 
    perguntando a um veterano que está fazendo a disciplina (o caminho mais 
    fácil)!
    \item \textbf{SINAPE}: o Simpósio Nacional de Probabilidade e Estatística 
    (SINAPE) é o maior e mais tradicional evento da área no Brasil, realizado a 
    cada dois anos pela Associação Brasileira de Estatística (ABE). A $26$ª 
    edição, por exemplo, ocorrerá em setembro de $2026$ em Gramado (RS). É um 
    ambiente que reúne pesquisadores internacionais, profissionais do mercado e 
    alunes de todo o país para debater Ciência de Dados, Probabilidade e métodos 
    quantitativos. Embora envolva custos de inscrição e viagem, é uma excelente 
    oportunidade para apresentar trabalhos e fazer contatos;
    \item \textbf{Escola de Modelos de Regressão (EMR)}: outro pilar da 
    Associação Brasileira de Estatística é a EMR, que ocorre a cada dois anos e 
    foca especificamente em modelagem de dados. Diferente do SINAPE, que é mais 
    abrangente, a EMR é um encontro sobre regressão (uma técnica estatística 
    muito utilizada), com minicursos e conferências de alto nível técnico. O 
    evento tem uma tradição de descentralização e a última edição foi em 
    Belém (PA).
\end{itemize}

\vspace{10pt}
\textbf{\Large Palavras Finais}
\vspace{5pt}

Entrar no Bacharelado em Estatística do IME-USP é aceitar um desafio que poucas 
pessoas têm coragem (ou paciência) de encarar. Você nem sempre gostará de todas 
as disciplinas, as listas de exercícios podem ser muitas e, às vezes, as dicas 
de veteranes podem ser seu único norte. Alguns dizem que é o curso mais difícil 
do IME. Mas, como você viu, a recompensa está à altura do esforço.

Seja você e bixe que acabou de descobrir o que é uma esperança condicional, ou e 
veterane que já está de olho no Doutorado ou naquela vaga exclusiva em uma super 
empresa, lembre-se: você não está sozinho. A força do nosso curso não está 
apenas no prestígio da universidade ou na qualidade do ensino e dos docentes, 
mas na rede que construímos aqui dentro, sejam nas monitorias, nos cafés da 
aMostra, nos grupos de WhatsApp ou nas conversas de corredor sobre qual matéria 
trancar ou contando algumas fofocas engraçadas também sobre docentes.

A Estatística é a ciência de medir a incerteza. Ironicamente, a única certeza 
que você pode ter ao sair pelo portão do IME com o diploma na mão é que o 
mercado e a academia estarão te esperando de braços abertos. Aproveite cada 
palestra no NUMEC, cada evento da ABE e cada projeto no CEA. O IME pode parecer 
bem difícil de sair, mas ele te transforma em um profissional muito bem 
qualificado e deixa muitas saudades pra quem se forma.

Nos vemos pelos corredores (ou em algum coffee break da aMostra)!

\end{subsecao}
