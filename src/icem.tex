\begin{subsecao}{iCEM}

\textbf{O que é?}

O iCEM é uma iniciativa pioneira do IME-USP, criada em 2024, para acolher estudantes ingressantes que tenham conquistado medalhas em olimpíadas de conhecimento (como OBM, OBMEP, OBI, OBF, entre outras) durante o ensino fundamental ou médio. O programa oferece apoio social e acadêmico logo nos dois primeiros anos da graduação, integrando estudantes de bacharelado e licenciatura ao universo da pesquisa científica desde o primeiro dia na universidade.

\textbf{Quais são os objetivos?}

O foco principal é criar uma comunidade de pertencimento e valorização do talento. 
Diferente das Iniciações Científicas tradicionais, que costumam começar nos últimos 
anos da graduação, o iCEM busca aproveitar o potencial precoce para: 

\begin{itemize} 
    \item Estimular o interesse pela carreira acadêmica e garantir a permanência 
    estudantil;
    \item Promover a diversidade e a representatividade feminina nas áreas de 
    ciências exatas (com a meta de alcançar 50\% de mulheres no programa); 
    \item Oferecer preparação para intercâmbio internacional e atividades de extensão; 
    \item Fortalecer a autoconfiança e o bem-estar emocional através de uma rede de 
    apoio sólida. 
\end{itemize} 

O reconhecimento da excelência do programa é internacional: em 2025, o iCEM foi um 
dos únicos programas da América Latina a receber o prêmio \textit{Google Academic 
Research Award}.

\textbf{Como realmente funciona?}

O iCEM funciona como uma rede integrada. Cada estudante participante é acompanhado 
por um docente (orientador acadêmico) e por um veterane do programa (tutor), criando 
um vínculo que auxilia tanto nas questões de pesquisa quanto na adaptação à vida u
niversitária.

Além desse suporte, o programa se estrutura através de parcerias robustas para o 
financiamento de bolsas, que vão além do Programa Unificado de Bolsas (PUB). O iCEM 
articula oportunidades através do PICME (Programa de Iniciação Científica e Mestrado), 
do Fundo Patrimonial da USP e da FAPESP, que recentemente criou a chamada ``Futuros 
Cientistas – Prof. Sérgio Muniz Oliva Filho'' especificamente para apoiar ingressantes 
com este perfil.

Se você tem histórico de medalhas, o IME quer reconhecer sua trajetória e oferecer as 
ferramentas para que você se torne e próximo grande cientista da nossa instituição.

Para mais informações e para demonstração de interesse em participar, acesse: 
\url{https://www.ime.usp.br/icem/}

\end{subsecao}
