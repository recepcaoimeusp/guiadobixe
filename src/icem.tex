\begin{subsecao}{iCEM}

\textbf{O que é?}

O iCEM (Iniciação Científica para Computação, Estatística e Matemática) é um
programa do IME que busca integrar estudantes dos bacharelados e licenciaturas
que tenham sido medalhistas em olimpíadas do conhecimento, já nos dois primeiros
anos de graduação.

\textbf{Quais são os objetivos?}

Entre os principais objetivos do programa estão a formação de uma base sólida, a
integração em comunidades acadêmicas e de extensão, o incentivo ao desenvolvimento
de habilidades de liderança e colaboração e, por fim, mas não menos importante, o
estímulo à participação em pesquisas com professores do IME.

\textbf{Como realmente funciona?}

O iCEM funciona como uma rede estruturada de apoio e integração entre ingressantes,
veteranos e docentes do IME. Cada estudante participante é acompanhado por um
professor orientador e por um veterano, que atuam, respectivamente, como orientador
acadêmico e monitor, auxiliando tanto em questões acadêmicas quanto na adaptação
à vida universitária.

Além desse acompanhamento, o iCEM oferece bolsas do Programa Unificado de Bolsas
(PUB), no valor aproximado de R\$ 700, e também auxilia os estudantes na busca por
outras oportunidades, como bolsas adicionais, projetos acadêmicos e possibilidades
de intercâmbio.

Para mais informações e para demonstração de interesse em participar, acesse:
\url{https://www.ime.usp.br/icem/}

\end{subsecao}
