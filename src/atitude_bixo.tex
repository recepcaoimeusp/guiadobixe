
\begin{secao}{Atitude, bixes!}
\quadrinhos{7}

Na USP, alunes têm a liberdade e apoio de se organizarem
para montar grupos de debates, ciclos de palestras, grupos
de desenvolvimento e até mesmo grupos para jogarem alguma coisa (como
RPG, Magic, Yu-Gi-Oh ou algum esporte).
Portanto, caso vocês tenham algum projeto em mente, não hesitem
em se organizarem com seus amigos e se informarem sobre como avançar com essa
ideia. Lembre-se de que seus veteranes estão aí para te aconselhar e tirar
eventuais dúvidas.

É possível também juntar-se com alguns amigos e formar grupos de
estudo, seja para alguma matéria com a qual vocês tenham dificuldade, para
discutir aquele EP/lista de exercícios que ninguém está conseguindo
fazer ou simplesmente estudar algum tópico de interesse mútuo.

A seguir, temos (majoritariamente) alguns dos exemplos dos grupos que foram 
direta ou indiretamente criados por alunes do nosso Instituto!

% RDs --------------------------------------------------------------------------
\begin{subsecao}{RDs}

Antes de mais nada, RD significa Representante Discente. O RD é o estudante 
que representa nossos interesses frente aos diversos conselhos e comissões 
institucionais existentes, sendo por vezes um elo de ligação entre professores 
e alunos. O RD atua na tomada de decisões internas do IME: mudanças no currículo, 
aumento de vagas na FUVEST e distribuição das cotas, reformas, mudança no corpo 
docente, enfim, coisas desse tipo.

Dá para perceber o quanto é importante ter um estudante em cada um desses conselhos 
visando o repasse e a transparência aos estudantes de questões administrativas do 
instituto. Então, ao assumir esse cargo, vocês poderão entender melhor o 
funcionamento do IME e, organizar coletivamente com o centro acadêmico, o processo de 
melhorá-lo.

Desde 2017, as eleições são organizadas anualmente pela administração do instituto 
ao final do segundo semestre. Anteriormente, as eleições eram organizadas pelo centro 
acadêmico, que criava uma Comissão Eleitoral independente e composta por membros do 
centro acadêmico e demais estudantes do IME interessados em organizar o processo eleitoral. 
Essa alteração permitiu a adoção de eleições on-line por meio do Helios Voting, e, embora 
haja o argumento de que os métodos virtuais sejam mais participativos, o quórum de votação 
nas eleições para RDs é quase metade do quórum das eleições das entidades representativas.

Bom, agora vejamos um breve resumo do que mais ou menos acontece em cada um dos 
colegiados nos quais temos direito a representante(s):

Existem diferentes níveis de hierarquia na administração.

{\bf As CoCs,
Comissões Coordenadoras de Curso (Lic, Pura, Estatística, Aplicada, BMAC e
Computação)} são as mais próximas dos alunos. Temos um cargo de aluno em cada
comissão. São comissões pequenas, que tratam dos problemas internos de cada
curso: mudança de currículo, requerimentos, optativas etc. São subordinadas
à CG e ao conselho do relativo departamento. Analogamente, temos um cargo em cada
Comissão Coordenadora de Programa (de Pós).

{\bf Os Conselhos de Departamento (MAT, MAE, MAC e MAP)} têm uma dinâmica um
pouco diferente das CoCs: são mais formais. Cada conselho se reúne (quase)
mensalmente e são formados (em geral) por mais pessoas, sendo que existem
regras sobre participação dos diferentes níveis hierárquicos de
professores (Titular, Associado, Doutor e Assistente). Nesses conselhos, além
de aprovar algumas das decisões das Comissões Coordenadoras de Curso e de
Programa (pós) e distribuição de carga didática, são discutidos reoferecimento
de curso, revisão de prova, supervisão das atividades dos docentes,
afastamentos (temporários ou não), contratação de professores e muitas outras
coisas.
Os Conselhos de Departamento são subordinados à Congregação e ao CTA.

{\bf A Comissão de Graduação (CG)}, basicamente, avalia requerimentos,
mudança/criação de cursos e jubilamentos. Analogamente, existe a Comissão de
Pós-Graduação (CPG). Ambas são subordinadas à Congregação.

{\bf A Comissão de Inclusão e Pertencimento (CIP)} lida com a questão de Permanência,
acompanhando a distribuição de auxílios feita pela reitoria, além de acompanhar
questões de saúde mental e eventualmente receber denúncias sobre opressões e 
violências que possam acontecer no instituto.

{\bf A Comissão de Cultura e Extensão (CCEx)} Cuida das atividades de extensão: 
Matemateca, CAEM, AACs que sejam extensão, etc.

Também há comissões mais específicas, como a comissão de estágio, a comissão de
pesquisa (do doutorado) e o Centro de Competência em Software Livre (CCSL), da
computação.

Os dois conselhos mais importantes são o CTA e a Congregação, ambos presididos
pelo Diretor.

{\bf O Conselho Técnico e Administrativo (CTA)} cuida de todas as questões não
acadêmicas: Orçamento, reformas, avaliação dos funcionários, Xerox etc. É
formado pelos quatro chefes de departamento, diretor, vice-diretor, um
representante dos funcionários e um RD.

{\bf A Congregação} é o órgão máximo do Instituto. Inclui muitos professores, a
maioria titular. São 3 RDs de graduação e 2 de Pós. Basicamente,
nesse órgão, são rediscutidas e aprovadas (ou não) muitas das decisões
dos órgãos subordinados. Os membros da Congregação têm voto na eleição para
Reitor e Vice-Reitor.

Bom, caso não tenha ficado claro desde o começo desse texto, percebam que é
muito importante ter um aluno em cada um desses conselhos. Se estiverem tendo
problemas com professores, requerimentos etc., ou simplesmente quiserem saber
o que anda acontecendo, procurem o RD certo pra conversar. Perguntem,
participem, votem e façam o IME um lugar melhor.

Sobre a eleição dos RDs: A eleição oficial para os RDs acontece no final do ano
(então fiquem atentos!) e é organizada pela diretoria do instituto. Os
interessados devem preencher um formulário de inscrição e levar até a
Assistência Acadêmica do IME, que vai organizar todos os inscritos e abrir um
processo de eleição on-line (em que todo IMEano pode votar). É indicado conversar
com os RDs anteriores para evitar que duas pessoas se inscrevam na mesma vaga e
alguma outra fique sem representação(se organizar direitinho, ocupamos todas as vagas
disponíveis). Quando a eleição
estiver se aproximando, vocês receberão (vários) e-mails com os documentos
necessários, prazos e links de votação.

%REFTIME
O resultado da eleição anterior com os RDs de 2025 pode ser encontrada no site
do CAMat:
\url{https://camat.ime.usp.br/}


\end{subsecao}


% Rede Linux -------------------------------------------------------------------
\input{rede_linux.tex}

% FLUSP ------------------------------------------------------------------------
\input{flusp.tex}

% IME Júnior -------------------------------------------------------------------
\input{ime_jr.tex}

% IME Finance ------------------------------------------------------------------
\input{ime_finance.tex}

% MaratonUSP -------------------------------------------------------------------
\begin{subsecao}{MaratonUSP}

\figurapequenainlineapertada{maratonusp}

O MaratonUSP é o grupo de programação competitiva da USP que tem como foco o
preparo de estudantes para a Maratona de Programação e Olimpíadas de Informática.

No início do ano, o grupo promove o BixeCamp que são aulas direcionadas aos bixes
para o estudo de algoritmos e estruturas de dados, por isso, não deixe de
participar delas. Assim, o MaratonUSP é, atualmente, referência para o
Brasil inteiro por suas diversas conquistas em competições e pelo seu canal do
YouTube, com aulas assistidas por milhares estudantes de programação competitiva
pelo país.

O ambiente de cooperação faz com que o grupo tenha resultados incríveis na
Maratona, sendo Tetracampeão da fase brasileira e tendo participado 16
vezes da etapa mundial da competição, em países como a China, Tailândia,
Rússia, Estados Unidos e muito mais - tudo de graça! Vale lembrar que a própria
fase nacional acontece em um local diferente do país todo ano, o que também gera
uma ótima oportunidade para conhecer mais lugares do Brasil.

Por fim, a vida de Maratoneire não impacta só sua passagem pela universidade,
ela também abre portas para o futuro. É bastante comum que ex-participantes
trabalhem em empresas como Google, Meta e Microsoft, tanto no Brasil,
quanto no exterior. As habilidades desenvolvidas durante as competições,
viajando pelo mundo, dando aulas, preparando vídeos para o YouTube ou
simplesmente estudando com colegas durante uma tarde serão úteis tanto no
mercado de trabalho quanto na academia, além de render anos de muita diversão!

\begin{description}
\item[Facebook:] \url{https://facebook.com/MaratonUSP}
\item[Site:] \url{https://www.ime.usp.br/~maratona}
\item[Youtube:] \url{https://www.youtube.com/c/maratonusp}
% Atualizar o link do Telegram todo ano (esse é específico de 2024)
\item[Telegram:] \url{https://t.me/bixes2025}
\end{description}

\end{subsecao}


% USPCodeLab -------------------------------------------------------------------
\begin{subsecao}{USPCodeLab}

\figurapequenainlineapertada{uspcodelab}

O USPCodeLab é um grupo de extensão que tem por objetivo criar um espaço
colaborativo para criar e incentivar o desenvolvimento de tecnologia na USP.

Nosso foco é aprender na prática ferramentas e técnicas de desenvolvimento de software que permitam
solucionar problemas do mundo real.

Durante o semestre, organizamos o webdev que é um curso dado por membros do codelab para ensinar o
básico de web (Html, Css e Javascript). Ao ganhar mais confiança no desenvolvimento web formamos
grupos de estudos para desenvolver projetos legais propostos pelos próprios participantes (básicos
e avançados). Alguns exemplos são um sistema de reserva de armários do CAMat e o grupo devboost que
desenvolveram um site para cadastrar oportunidades (empregos, IC, \dots). 

O USPCodeLab também organiza hackatons como o shehacks e o hackfools, estes são competições de
programação que os participantes se dividem em grupos para pensar na solução de um problema e tentar
elaborar.

Curtam nossa página do Facebook e entrem no nosso grupo do Telegram para saber
datas e horários das nossas reuniões abertas. Participem do USPCodeLab!

\begin{description}
\item[Facebook:] \url{https://uclab.xyz/facebook}
\item[Telegram:] \url{https://t.me/codelabbutanta}
\item[Site:] \url{https://uclab.xyz/site}
\item[Instagram:] @uspcodelab
\end{description}

\end{subsecao}


% USPGameDev: Pesquisa e Desenvolvimentos de Jogos na USP ----------------------
\input{usp_game_dev.tex}

% IMEsec -----------------------------------------------------------------------
\filbreak
\begin{subsecao}{IMEsec}

\figurapequenainline{imesec}

O IMEsec é um grupo de extensão focado em aprender, estudar e se divertir com a
segurança da informação. Sem que a maioria das pessoas se dê conta, este nicho
está presente no cotidiano. Mandar uma mensagem no WhatsApp, navegar pela web,
entrar no Facebook: exemplos de ações simples do dia-a-dia que necessitam ser
feitas de maneira segura, visando à privacidade do usuário.

Nosso grupo, formado majoritariamente por alunos do IME-USP no início de 2017,
busca entender melhor este universo e expandi-lo no ambiente universitário. O
foco tem sido amplo; desde resolução de desafios on-line (que são muito
divertidos), participação em competições (sim, competições -- \textit{capture the
flag} — muito, muito legais), apresentação de palestras e até desenvolvimento de
projetos que possam beneficiar a população. Todos são bem-vindos; basta ter
interesse pelo assunto.

As experiências decorrentes de nossas atividades ajudaram e ajudam nos estudos,
além de agregarem valor à graduação. Por meio desses, mergulhamos não só em
computação, mas também em matemática, estatística e até outras áreas bem
inusitadas.

Fiquem de olho no nosso grupo do Telegram, ou no nosso servidor do Discord, 
para saber mais sobre as reuniões semanais, além de fatos interessantes (ou 
simplesmente engraçados) sobre o vasto mundo da segurança. Participem!

\begin{description}
  \item[Discord:] \url{https://discord.gg/ZcMvXStFVS}
  \item[Telegram:] \url{https://tiny.cc/imesec-telegram}
  \item[Site:] \url{https://imesec.ime.usp.br/}
\end{description}

\end{subsecao}


% L.E.A.R.N. -----------------------------------------------------------------------
\input{learn.tex}

% SymComp -----------------------------------------------------------------------
\input{symcomp.tex}

% Hardware Livre ---------------------------------------------------------------
\input{hardware_livre.tex}

% Tecs: Computação Social ------------------------------------------------------

\begin{subsecao}{Tecs}

\figurapequenainline{tecs}

Nós somos um grupo de extensão focado no impacto social da computação e da 
tecnologia e que desenvolve projetos em três frentes: educação, ética e 
serviços. Nossos projetos visam, respectivamente, promover a educação tecnológica 
igualitária da população por meio de cursos, oficinas e ações promovidas pelo 
grupo; unir esforços para formar uma sociedade e profissionais éticos e
conscientes sobre o uso da tecnologia; e estimular alunos a usarem a tecnologia 
para solucionar problemas da comunidade local.

Visamos a um cenário de equidade dos saberes, no qual as universidades superem
as barreiras de restrição de conhecimentos e técnicas. Pretendemos, a longo prazo,
contribuir para uma formação universitária que estimule maior consciência social, 
capaz de gerar profissionais da área de computação hábeis em refletir sobre as 
implicações éticas e sociais do seu trabalho, desmentindo, assim, o mito da 
neutralidade tecnológica.

Em termos gerais, pretendemos que os estudantes entendam como a tecnologia pode
ser utilizada para o bem coletivo, e utilizem esse conhecimento na prática, por
meio de colaborações com a comunidade local, os serviços públicos, as
organizações não-governamentais e as sem fins econômicos. 

Se você tem interesse em promover o ensino de computação, em estudar e debater 
questões éticas e sociais no contexto tecnológico, ou em desenvolver aplicativos,
sites ou sistemas em parcerias com projetos sociais, entre em contato conosco e 
participe do grupo!

\vspace{-1em}
\begin{description}
  \item[Site:] \url{www.ime.usp.br/~tecs}
  \item[Facebook:] \url{www.facebook.com/tecs.usp}
  \item[Telegram:] \url{https://t.me/tecsusp}
  \item[Twitter:] \url{https://twitter.com/tecsusp}
  \item[Instagram:] \url{https://www.instagram.com/tecs.usp} 
\end{description}

\end{subsecao}


% DynaUSP: Empreendedorismo
\input{dynausp.tex}

% Diversime --------------------------------------------------------------------

\input{diversime.tex}

% Existimos! -------------------------------------------------------------------
\input{existimos.tex}

% Comissão de Acolhimento da Mulher! -------------------------------------------
\input{cam.tex}

% Comissão de Inclusão e Pertencimento -----------------------------------------
\begin{subsecao}{Comissão de Inclusão e Pertencimento- CAM}

O que é a CIP?

A Comissão de Inclusão e Pertencimento do IME-USP é composta 
por pessoas de diferentes setores do IME e tem por objetivo promover e 
zelar por um ambiente que valorize e acolha a pluralidade através de atividades 
relacionadas a inclusão, pertencimento, diversidade e equidade no âmbito do IME.

Quando procurar a CIP ❓

- Solicitar adaptações pedagógicas ou funcionais para pessoas com TEA e/ou outras condições neurodivergentes.
- Buscar ajuda ligada a questões de saúde mental, seja para você ou para outra pessoa do IME.
- Buscar acolhimento ou denunciar comentários preconceituosos e atitudes inadequadas nos espaços do IME.
- Procure a CIP mesmo em caso de dúvida. Se a sua demanda não for de nossa competência, podemos ajudar a encaminhar para outras instâncias.

Onde nos encontrar?

Atendimento presencial:
Sala 260, Bloco A do IME-USP
(2º andar – em frente ao elevador)

Horário de atendimento:
Das 13:00 às 19:00, de segunda a sexta

Conheça a gestão: 

\textbf{Professoras}: 
\vspace{-15pt}
\begin{itemize}
  \item Cristina Brech ({\tt brech@ime.usp.br})
  \item Renata Wassermann ({\tt renata@ime.usp.br})
\end{itemize}

Para mais informações, envie um e-mail para {\tt cip@ime.usp.br} ou acesse:
\begin{itemize}
  \item Site: \url{https:www.ime.usp.br/cip}
\end{itemize}


\end{subsecao}

% CinIME -----------------------------------------------------------------------
\filbreak
\begin{subsecao}{CinIME}

\figurapequenainline{novo-logo-cinime}

Convidamos vocês a conhecer o CinIME!! Somos um projeto do CAMat que promove a
exibição de filmes para discentes, docentes e funcionários do IME. Queremos pautar
de forma mais crítica o audiovisual, oferecendo um momento reflexivo, mas que 
também seja de lazer e diversão.

O CinIME ocorre toda sexta-feira. A sessão, o refrigerante e a pipoca são de graça. 
A organização é feita a partir de uma comissão aberta ao público, que
qualquer estudante pode compor e trabalhar conjuntamente na construção do projeto. 

Além das sessões usuais, também fazemos projetos paralelos, como a MostrIME - Mostra de
Animação no IME, em que exibimos dezenas de filmes de animação durante as férias, o Corujão
Satoshi Kon, que trouxe três filmes desse diretor exibidos numa madrugada, no CCSL (sim, a gente
passou a madrugada no IME vendo filmes!) e a nossa exibição do Oscar, em que escolhemos um filme
não-indicado para homenagearmos. 

A comissão organizadora do CinIME atualmente é guiada pela seguinte questão: "Quais 
horizontes imaginativos o cinema pode proporcionar aos estudantes do IME?". Mais detalhes
sobre isso podem ser encontrados no nosso projeto, o documento mais importante sobre o 
CinIME, que está disponível no site do CAMat.

Não se esqueçam de sugerir filmes, votar e comparecer ao CinIME!

Sigam as redes do CAMat, entrem na nossa comunidade no Discord e fiquem por dentro das novidades!

\begin{description}
  \item[Discord:] \url{https://discord.gg/qDfXUMVm6j}
  \item[Site:] \url{https://camat.ime.usp.br/cinime/}
  \item[Instagram:] \url{https://www.instagram.com/camat.usp/}
\end{description}

\end{subsecao}


% Grupo A5 ---------------------------------------------------------------------
\begin{subsecao}{Grupo S4}

\begin{wrapfigure}{r}{0.25\textwidth}
    \vspace{-25pt}
    \begin{center}
      \includegraphics[width=0.25\textwidth]{img/logo S4 maior.png}
    \end{center}
    \vspace{-25pt}
  \end{wrapfigure}


Nós somos um grupo de estudantes de diferentes anos da graduação e pós-graduação 
do IME, e organizamos seminários feitos por alunes e para alunes em diversos
tópicos de matemática, muitos não vistos na graduação. Originado em 2015 na forma
de seminários de iniciação científica, o S4 tem hoje em dia o propósito de apresentar seminários 
periódicos acessíveis e interessantes para estudantes que ainda não possuem muita 
bagagem matemática, inclusive de outros cursos da USP. 

Somos independentes de outras instituições do IME, e estamos sempre abertos a 
receber novos participantes, dos mais engajados aos mais casuais. É um espaço
para que es alunes tenham a oportunidade de apresentar algum tema que tenham
estudado, e começar na prática a treinar a habilidade de comunicação científica. 
Es alunes, independentes do ano, estão convidades a sugerir temas e dar ideias
para a organização de seminários. Caso desejem apresentar um seminário, nós
poderemos ajudar no seu preparo em um ambiente amistoso e receptivo.

Além disso, o S4 também é um ambiente de estudo e socialização, onde todes podem
compartilhar dúvidas, sugestões, conhecer novos bixes e veteranes, bater papo,
descontrair, etc. Estamos sempre dispostos a ajudar quaisquer alunes, especialmente
bixes, para que possamos ver um pouco mais do que a matemática tem a oferecer e
compartilhar as diferentes visões sobre ela de cada um de nós.

\begin{description}
  \item[Discord:] \url{discord.gg/XzXeCzc3rk}
  \item[E-mail:] \url{grupo-s4@riseup.net}
\end{description}

\end{subsecao}


% Olimpíadas de Matemática e Informática ---------------------------------------
% Maratona de Programação ------------------------------------------------------
\begin{subsecao}{Olimpíadas de Conhecimento}

\begin{itemize}

\item{\bf Matemática: }

Bom, pessoal, se vocês entraram no IME, podem ter ouvido falar ou participado
de alguma Olimpíada de Matemática no Ensino Fundamental e/ou Médio. A
boa notícia é que vocês vão poder continuar participando se quiserem,
e quem nunca participou tem a oportunidade de começar agora.

Mas por que participar? As Olimpíadas Universitárias de Matemática são uma
oportunidade de se divertir resolvendo problemas difíceis de Matemática e agregar
valor ao currículo ao mesmo tempo. Elas são parecidas com as Olimpíadas de
Ensino Médio, mas com conteúdo de Matemática da graduação (essencialmente
Cálculo, Análise, Álgebra Linear, Álgebra, Combinatória e Teoria dos Números),
mas com enfoque em problemas que exigem criatividade e técnicas mais inovadoras,
muitas das quais vocês provavelmente não verão durante toda a graduação.

De quais olimpíadas podemos participar? Como alunos de graduação, vocês podem
participar da Olimpíada Iberoamericana de Matemática Universitária (OIMU),
Olimpíada Brasileira de Matemática (OBM) e Olimpíada Internacional de
Matemática (IMC).

Como fazemos para nos preparar? Os sites institucionais dessas olimpíadas
têm material voltado para vocês que querem estudar e se preparar, mas vocês
podem procurar alguém mais experiente para indicar alguma bibliografia por fora
também. 

Como fazemos para participar? Inscrevam-se pelo site ou entrem em contato com
o professor Colucci. Para o IMC aconselha-se ter ganhado medalha na OBM,
já que é necessário apoio financeiro do IME por ser uma olimpíada internacional.

%REFTIME
Mas nós, bixes, temos chance? Como foi o desempenho de IMEanos nelas? A organização 
da OBM criou a Copa Elon Lages Lima, que é a primeira fase da OBM, mas tem uma 
premiação própria e um nível mais acessível para quem chegou agora. Desde 2021, 
vários alunos do IME conseguiram medalhas, inclusive bixes que foram chamados para participar 
da OBM! É a melhor porta de entrada para quem ainda não tem muita experiência. 

Se tiverem alguma dúvida, não hesitem em perguntar a algum veterane sobre os
Olímpicos!

Links institucionais:

\begin{description}
  \item[] \url{http://oimu.eventos.cimat.mx}
  \item[] \url{https://www.imc-math.org.uk/}
  \item[] \url{http://www.obm.org.br}
\end{description}

\item{\bf Informática: }

\textit{``Informática? Vocês mexem com Word, Excel e PowerPoint então?''}

Responder essa pergunta já virou rotina para competidores da Olimpíada
Brasileira de Informática (OBI). Não, Informática não é Word. Oras, então o que é a OBI?

A OBI é uma competição de lógica, matemática e computação. As provas envolvem
alguns problemas que você deve resolver com programas de computador.

Esta competição, na graduação, é exclusiva para ingressantes recém formados do
ensino médio. Quer dizer que vocês são a nossa única esperança de trazer mais
gloriosas medalhas ao IME! Isso também quer dizer que essa é a sua única chance
de participar da OBI, uma competição relativamente tranquila comparada à
Maratona de Programação.

Para participar, basta falar com o MaratonUSP
(\url{https://www.ime.usp.br/~maratona/}), um grupo de extensão focado nesse
tipo de competição que promete te ajudar a se inscrever e se preparar, ou com o
Professor Carlinhos (\url{http://www.ime.usp.br/~cef/}).

%\item{\bf Maratona de Programação: }

%À primeira vista, a Maratona de Programação pode soar um tanto
%surreal. Nerds correndo pela USP ao mesmo tempo que resolvem
%problemas de computação e matemática? Infelizmente esse não
%é o caso.

%A Maratona de Programação se resume à resolução de problemas.
%Se você adora resolver desafios, quebrar a cabeça com novos
%e excitantes problemas e acumular toneladas de dinheiro, esse
%é o lugar perfeito para você!

%A competição consiste em uma série de problemas que englobam
%temas como programação dinâmica, grafos e estruturas de dados.
%Times de três pessoas devem resolver a maior quantidade de
%desafios em cinco horas de programação. E tudo isso com direito
%a um lanche gratuito durante a prova.

%Mas não temam, bixes. Não é só por que vocês acabaram de entrar que
%a probabilidade de ganhar uma medalha seja nula. Inclusive, na primeira
%fase da maratona, uma equipe de bixes tem vaga garantida para a
%fase brasileira.

%Além da fama, constantes pedidos por autógrafos e dinheiro de sobra,
%a maratona também vai lhes trazer um conhecimento muito mais
%adiantado em relação ao dos seus colegas de classe, e até oportunidades
%de emprego em empresas de renome, como Google, Facebook e IBM.

%Se vocês se interessaram pela maratona e querem saber os horários dos
%treinos, como participar ou saber mais, acessem:

%\begin{description}
%  \item[Site:] \url{http://www.ime.usp.br/~maratona}
%  \item[Site da competição nacional:] \url{http://maratona.ime.usp.br/}
%\end{description}

\end{itemize}


\end{subsecao}


% Fala Sério -------------------------------------------------------------------
%\input{fala_serio.tex}

% Bee Data USP -----------------------------------------------------------------
\input{beedata.tex}

% Coloquinho da Pós-Graduação ---------------------------------------------------
\begin{subsecao}{Coloquinho da Pós-Graduação}

\figurapequenainline{coloquinho}

O Coloquinho da Pós-Graduação foi criado para proporcionar um espaço de discussão 
acadêmica e troca de conhecimentos entre os pós-graduandos e o público geral. 
As apresentações orais abrangem uma ampla variedade de tópicos relevantes da 
Matemática Pura e Aplicada. 

No 1° semestre de 2025, os seminários serão às \textbf{terças-feiras} às
\textbf{13:00}, no IME-USP. O primeiro seminário deste ano ocorrerá no 
dia 19/03/2025. 

Todes são bem-vindes para assistir os seminários. Emitimos certificados 
de participação para as horas complementares. 

Acessem o nosso site e nos sigam no Instagram para verem as apresentações
anteriores e acompanharem as atualizações sobre as datas e o local dos próximos 
seminários!

\textbf{Instagram:} @coloquinho.ime
\textbf{Email:} @coloquinhoime@gmail.com
\textbf{Site:} https://sites.google.com/ime.usp.br/coloquinho 



\end{secao}
