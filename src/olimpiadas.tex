\begin{subsecao}{Olimpíadas de Conhecimento}

\begin{itemize}

\item{\bf Matemática: }

Bom, pessoal, se vocês entraram no IME, podem ter ouvido falar ou participado
de alguma Olimpíada de Matemática no Ensino Fundamental e/ou Médio. A
boa notícia é que vocês vão poder continuar participando se quiserem,
e quem nunca participou tem a oportunidade de começar agora.

Mas por que participar? As Olimpíadas Universitárias de Matemática são uma
oportunidade de se divertir resolvendo problemas difíceis de Matemática e agregar
valor ao currículo ao mesmo tempo. Elas são parecidas com as Olimpíadas de
Ensino Médio, mas com conteúdo de Matemática da graduação (essencialmente
Cálculo, Análise, Álgebra Linear, Álgebra, Combinatória e Teoria dos Números),
mas com enfoque em problemas que exigem criatividade e técnicas mais inovadoras,
muitas das quais vocês provavelmente não verão durante toda a graduação.

De quais olimpíadas podemos participar? Como alunos de graduação, vocês podem
participar da Olimpíada Iberoamericana de Matemática Universitária (OIMU),
Olimpíada Brasileira de Matemática (OBM) e Olimpíada Internacional de
Matemática (IMC).

Como fazemos para nos preparar? Os sites institucionais dessas olimpíadas
têm material voltado para vocês que querem estudar e se preparar, mas vocês
podem procurar alguém mais experiente para indicar alguma bibliografia por fora. 
Também existe o USP Olympic Math Circle (\url{https://www.instagram.com/usp.omc/}), 
um grupo de estudos, treinamento e divulgação das Olimpíadas Universitárias, 
com reuniões semanais no IME.

Como fazemos para participar? Inscrevam-se pelo site ou entrem em contato com
o professor Colucci. Para o IMC aconselha-se ter ganhado medalha na OBM,
já que é necessário apoio financeiro do IME por ser uma olimpíada internacional.

%REFTIME
Mas nós, bixes, temos chance? Como foi o desempenho de IMEanos nelas? A organização 
da OBM criou a Copa Elon Lages Lima, que é a primeira fase da OBM, mas tem uma 
premiação própria e um nível mais acessível para quem chegou agora. Desde 2021, 
vários alunos do IME conseguiram medalhas, inclusive bixes que foram chamados para participar 
da OBM! É a melhor porta de entrada para quem ainda não tem muita experiência. 

Se tiverem alguma dúvida, não hesitem em perguntar a algum veterane sobre os
Olímpicos!

Links institucionais:

\begin{description}
  \item[] \url{http://oimu.eventos.cimat.mx}
  \item[] \url{https://www.imc-math.org.uk/}
  \item[] \url{http://www.obm.org.br}
\end{description}

\item{\bf Informática: }

\textit{``Informática? Vocês mexem com Word, Excel e PowerPoint então?''}

Responder essa pergunta já virou rotina para competidores da Olimpíada
Brasileira de Informática (OBI). Não, Informática não é Word. Oras, então o que é a OBI?

A OBI é uma competição de lógica, matemática e computação. As provas envolvem
alguns problemas que você deve resolver com programas de computador.

Esta competição, na graduação, é exclusiva para ingressantes recém formados do
ensino médio. Quer dizer que vocês são a nossa única esperança de trazer mais
gloriosas medalhas ao IME! Isso também quer dizer que essa é a sua única chance
de participar da OBI, uma competição relativamente tranquila comparada à
Maratona de Programação.

Para participar, basta falar com o MaratonUSP
(\url{https://www.ime.usp.br/~maratona/}), um grupo de extensão focado nesse
tipo de competição que promete te ajudar a se inscrever e se preparar, ou com o
Professor Carlinhos (\url{http://www.ime.usp.br/~cef/}).

%\item{\bf Maratona de Programação: }

%À primeira vista, a Maratona de Programação pode soar um tanto
%surreal. Nerds correndo pela USP ao mesmo tempo que resolvem
%problemas de computação e matemática? Infelizmente esse não
%é o caso.

%A Maratona de Programação se resume à resolução de problemas.
%Se você adora resolver desafios, quebrar a cabeça com novos
%e excitantes problemas e acumular toneladas de dinheiro, esse
%é o lugar perfeito para você!

%A competição consiste em uma série de problemas que englobam
%temas como programação dinâmica, grafos e estruturas de dados.
%Times de três pessoas devem resolver a maior quantidade de
%desafios em cinco horas de programação. E tudo isso com direito
%a um lanche gratuito durante a prova.

%Mas não temam, bixes. Não é só por que vocês acabaram de entrar que
%a probabilidade de ganhar uma medalha seja nula. Inclusive, na primeira
%fase da maratona, uma equipe de bixes tem vaga garantida para a
%fase brasileira.

%Além da fama, constantes pedidos por autógrafos e dinheiro de sobra,
%a maratona também vai lhes trazer um conhecimento muito mais
%adiantado em relação ao dos seus colegas de classe, e até oportunidades
%de emprego em empresas de renome, como Google, Facebook e IBM.

%Se vocês se interessaram pela maratona e querem saber os horários dos
%treinos, como participar ou saber mais, acessem:

%\begin{description}
%  \item[Site:] \url{http://www.ime.usp.br/~maratona}
%  \item[Site da competição nacional:] \url{http://maratona.ime.usp.br/}
%\end{description}

\end{itemize}


\end{subsecao}
