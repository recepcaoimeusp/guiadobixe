\begin{secao}{Dicas}

Como vocês sempre chegam perdides, aqui vão algumas dicas pra vocês não
ficarem perguntando o tempo todo:

{\bf Bancos e Caixas Eletrônicos:} agências do Santander, Bradesco,
Banco do Brasil, Itaú e Caixa Econômica na Av. Prof. Luciano Gualberto.

{\bf MatrUSP:} é um site criado por alunos do BCC para simular grades horárias
para matrículas de disciplinas. Assim, vocês conseguem saber direitinho se suas
disciplinas vão coincidir e quanto tempo livre vocês vão ter para ficar na
vivência entre as aulas. Acessem: \url{https://bcc.ime.usp.br/matrusp}

{\bf Aurora:} mais um site criado por alunos do BCC, lançado em 2025, para saber o seu progresso na graduação. Muito bom para gerenciar os seus créditos já feitos e os que ainda faltam, além das disciplina que irão ser cursadas em cada semestre. Ele salva todo o seu planejamento na sua conta, mantendo ele consistente durante toda a sua graduação! Acessem: \url{https://aurora.ime.usp.br}

{\bf USPAvalia:} outro site criado por alunos do BCC (pois é, olha só!) que
contém inúmeras avaliações da comunidade sobre disciplinas oferecidas e
seus respectivos professores. O site não tem sido mantido, então faltam vários
professores (especialmente os mais novos), mas pode ser útil para saber se vale a pena 
ou não pegar essa ou aquela turma de uma dada disciplina. Vocês também podem 
contribuir com suas próprias avaliações e comentários! Acessem: \url{https://uspavalia.com}

{\bf Banco de Provas do CAMat:} uma pasta compartilhada com várias provas de
anos anteriores, que é muuuito boa para estudar. Quando estiver chegando aquela
prova daquela matéria difícil, vale a pena dar uma olhada se tem uma prova antiga,
e depois de passar (ou não) também vale colocar a prova que você fez pra ajudar
as próximas turmas! Acessem: \url{https://tinyurl.com/provas-camat}

\begin{subsecao}{Cultura na USP}

Importante lembrar que a entrada em vários desses museus é gratuita para quem é da
USP, então vale muito a pena visitar!

{\bf Museu de Arqueologia e Etnologia (MAE):} ao lado da Prefeitura do Campus,
possui um dos maiores acervos de artefatos arqueológicos e etnográficos do Brasil.

{\bf Museu de Arte Contemporânea (MAC):} localizado fora da USP, no Parque Ibirapuera, 
nele são expostas produções artísticas nacionais e estrangeiras.

{\bf Museu da Educação e do Brinquedo (MEB):} fica na Faculdade de Educação, Bloco B. 
Seu acervo conta 
com itens datados do início do século XX, incluindo brinquedos, jogos, materiais
pedagógicos e um acervo fotográfico. Tem a exposição ``Cenas Infantis'', que fica na
biblioteca da Faculdade de Educação.

{\bf Museu do Crime da Polícia Civil:} na Academia de Polícia perto do P1. Seu acervo
reúne ferramentas, objetos e documentos utilizados em delitos de grande repercussão, 
além da história de criminosos cujos atos ficaram marcados na imprensa e na sociedade
brasileira. Tal como o famoso crime da mala, ocorrido em 1928.

{\bf Museu do Instituto Oceanográfico:} adivinha? Esse museu mantém uma exposição
voltada à dinâmica, à estrutura e à biodiversidade dos oceanos. 

{\bf Museu de Geociências:} lá mesmo. Conta com amostras de rochas, gemas, meteoritos,
fósseis e venda de cristais. Tecnicamente não faz parte do museu, mas o instituto tem um 
esqueleto de Tiranossauro Rex no saguão do térreo.

{\bf Instituto Butantan:} próximo à História. É um dos maiores acervos de pesquisa
biológica do mundo, conta com a presença de diversas cobrinhas. 

{\bf Museu de Anatomia Veterinária:} perto do P3 (portão da Corifeu). Seu acervo
possui uma coleção de dados e fotos de esqueletos, além de modelos anatômicos e
animais preservados.

{\bf Museu de Anatomia Humana:} do lado do HU, seu acervo conta com inúmeras peças
de partes do corpo humano, reais e preservadas, além de modelos anatômicos.

{\bf Orquestra Sinfônica da USP (OSUSP):} faz ensaios abertos no Anfiteatro
Camargo Guarnieri, perto do CRUSP.

{\bf Teatro da USP (TUSP):} fora da USP e próximo do Mackenzie (estação
Higienópolis-Mackenzie), o teatro conta com apresentações frequentes e às vezes
um processo seletivo no final do ano.

{\bf CoralUSP:} realiza apresentações em vários locais de São Paulo, mas geralmente
pode ser encontrado no Anfiteatro Camargo Guarnieri, perto do CRUSP. Ocasionalmente
abrem inscrições.

{\bf Museu Paulista, vulgo Ipiranga:} também fora da USP, no Parque da Independência, 
possui exposições sobre a história do Brasil e é um dos museus mais famosos da cidade.

{\bf Museu de Zoologia:} novamente fora da USP, na Av. Nazaré, 481 -
Ipiranga. Sua exposição abriga uma série de animais empalhados, fósseis,
réplica de fósseis etc.

{\bf Museu Histórico Professor Carlos da Silva Lacaz (FMUSP):} fora da cidade universitária e dentro 
da faculdade de medicina da USP, perto da estação Clínicas, o museu atua como um centro de 
preservação, pesquisa e divulgação da história da medicina e das práticas da saúde. Oferece 
visitas que podem ser guiadas e serve como fonte para pesquisas acadêmicas. Há uma série de 
aparelhos ee instrumentos médicos antigos, mobiliário da época, vestuários, condecorações, 
documentos históricos, fotografias e peças de ceroplastia.

{\bf Estação Ciência da USP:} fechada ao público e com suas atividades encerradas 
no histórico galpão da Lapa, porém, funcionava como centro de divulgação científica 
pautado na interatividade. Oferecia uma experiência prática onde a ciência ganhava 
vida, com simuladores de fenômenos naturais, geradores de eletroestática e 
experimentos de óptica e mecânica. Após o encerramento das operações, grande parte 
do patrimônio educativo e dos equipamentos foi redistribuída para outros orgãos da 
universidade, como o Parque CienTec e o Museu de Zoologia.

{\bf Espaço das Artes (EDA):} espaço de criação e experimentação artística destinado 
ao ensino, à pesquisa e à extensão da ECA. Serve como um grande laboratório para alunos, 
docentes e pesquisadores dos três departamentos de artes da ECA (artes cênicas, artes plásticas e música) 
e dos programas de pós-graduação.

{\bf CinUSP Paulo Emílio:} dentro do campus, próximo ao bandejão Central,
existe uma sala de cinema. Durante todo o ano ocorrem várias mostras 
cinematográficas tematizadas, normalmente por mês, nas quais são exibidos inúmeros filmes. 
As sessões são gratuitas (!!!) e a programação pode ser conferida no seguinte site: \url{https://www.usp.br/cinusp/}

{\bf Centro de Divulgação Científica e Cultural - São Carlos (CDCC):} Sediado 
em um edifício histórico de 1908 no centro de São Carlos, o Centro de Divulgação 
Científica e Cultural (CDCC) funciona como um polo interativo de educação não formal, 
dedicado a aproximar a comunidade dos conceitos científicos. O que ele tem à mostra 
inclui a "Sala de Eletricidade" com experimentos físicos demonstrativos, o 
"Jardim da Percepção" que estimula os sentidos para compreender fenômenos naturais, 
além de exposições temporárias sobre a fauna local e um Observatório Astronômico equipado 
com telescópios para observação do céu noturno e do Sol.

{\bf Museu e Centro de Ciências, Educação e Artes "Luiz de Queiroz" - Piracicaba:} Localizado no 
campus da ESALQ em Piracicaba, o Museu e Centro de Ciências, Educação e Artes "Luiz de Queiroz" 
atua na preservação da memória da instituição e de seu fundador, funcionando em um casarão 
histórico cercado por jardins. Seu acervo à mostra é composto por salas temáticas que narram 
a história do café e da agricultura, mobiliário de época, objetos pessoais de Luiz de Queiroz, 
instrumentos científicos antigos e obras do artista Renato Wagner, oferecendo uma visão da 
evolução do ensino agronômico e da vida no campus.

{\bf Museu Republicano Convenção de Itu - Itu:} Situado na histórica cidade de Itu, 
o Museu Republicano Convenção de Itu é uma extensão do Museu Paulista da USP e funciona 
no sobrado onde ocorreu a famosa Convenção de 1873, marco do movimento republicano. 
O museu apresenta exposições focadas na história do Brasil entre os séculos XIX e XX, 
tendo à mostra um rico acervo de azulejaria, mobiliário, louças, fotografias e documentos 
que retratam o cotidiano e a cultura material da sociedade paulista e a formação do regime 
republicano.

{\bf Ruínas Engenho São Jorge dos Erasmos - Santos:} Como um museu a céu aberto na divisa 
entre Santos e São Vicente, as Ruínas Engenho São Jorge dos Erasmos funcionam como um sítio 
arqueológico e base de pesquisa sobre a colonização portuguesa e a indústria açucareira do 
século XVI. O espaço oferece trilhas educativas onde estão à mostra as ruínas do engenho de 
açúcar mais antigo do Brasil (fundado em 1534), permitindo aos visitantes observar as estruturas 
de pedra originais, vestígios arqueológicos e compreender a dinâmica social e econômica do 
período colonial.

{\bf ESALQ-LOG - Museu de Logística - Piracicaba:} Anexo ao grupo de pesquisa ESALQ-LOG 
em Piracicaba, o Museu de Logística opera como um espaço de preservação da história do 
transporte e da movimentação de cargas, especialmente voltado ao agronegócio. O que ele 
tem à mostra, distribuído em uma área aberta, inclui peças de grande porte como uma locomotiva, 
vagões ferroviários, caminhões antigos, uma barcaça e contêineres, ilustrando a evolução dos 
diferentes modais logísticos utilizados no escoamento da produção agrícola.

{\bf Orquestra de Câmara da USP - São Paulo (OCAM):} Sediada no Departamento de Música da 
ECA-USP, a OCAM funciona como um laboratório artístico de excelência voltado para a formação 
profissional de jovens músicos. Sob a regência do maestro Gil Jardim, a orquestra apresenta 
temporadas regulares de concertos e espetáculos inovadores, tendo à mostra um repertório 
abrangente que dialoga com a tradição erudita e a música popular brasileira, promovendo 
a integração entre a universidade e a sociedade.

{\bf USP Filarmônica - Ribeirão Preto:} Vinculada à Faculdade de Filosofia, Ciências 
e Letras de Ribeirão Preto (FFCLRP), a USP Filarmônica atua como uma orquestra acadêmica 
formada exclusivamente por alunos de graduação bolsistas. O que ela oferece ao público são 
séries de concertos gratuitos em Ribeirão Preto e cidades vizinhas, funcionando como um 
pilar de extensão universitária que une o ensino musical de alta performance à divulgação 
de obras sinfônicas clássicas e contemporâneas.

{\bf Edusp (Editora da USP):} Como editora oficial da universidade, a Edusp funciona publicando 
obras de relevância acadêmica, científica e cultural, consolidando-se como uma das principais 
casas editoriais do país. O que ela tem à mostra em seu catálogo e em suas oito livrarias físicas 
(espalhadas pelos campi da capital e do interior) é uma vasta seleção de títulos que vão desde 
clássicos da literatura e filosofia até pesquisas de ponta produzidas na USP, garantindo o acesso 
democrático ao conhecimento produzido na academia.

{\bf Centro de Preservação Cultural (CPC) / Casa de Dona Yayá - São Paulo:} Sediado no bairro da 
Bela Vista (Bixiga), o Centro de Preservação Cultural (CPC) / Casa de Dona Yayá funciona em uma 
histórica residência tombada que serve como polo de reflexão sobre patrimônio cultural e memória 
social. O que ele tem à mostra é a própria arquitetura eclética do casarão e seu jardim, que 
testemunham as transformações urbanas de São Paulo e a trágica história de sua antiga proprietária, 
além de exposições temporárias, cursos e apresentações musicais que ocupam o espaço.

{\bf Centro Universitário Maria Antonia - São Paulo:} Ocupando os históricos edifícios da antiga 
Faculdade de Filosofia, Ciências e Letras da Universidade de São Paulo (FFCL-USP) na Vila Buarque, 
o Centro Universitário Maria Antonia funciona como um espaço de efervescência cultural, arte e 
debate político, mantendo viva a memória da resistência democrática e do movimento estudantil. 
O que ele tem à mostra inclui uma programação diversificada de exposições de arte contemporânea 
e fotografia, além de oferecer cursos de difusão, biblioteca, cinema e teatro abertos à comunidade.

{\bf Centro de Educação Física, Esportes e Recreação - Ribeirão Preto (CEFER):} Localizado no campus 
de Ribeirão Preto, o CEFER-RP (Centro de Educação Física, Esportes e Recreação) funciona como um complexo 
esportivo destinado ao bem-estar da comunidade universitária. Suas instalações, que incluem ginásio 
poliesportivo, quadras de tênis, campo de futebol, pista de atletismo e piscinas, estão à mostra 
para uso de alunos e servidores, oferecendo programas de práticas esportivas orientadas e espaços 
de lazer para integração social.

{\bf Centro de Educação Física, Esportes e Recreação - São Carlos (CEFER):} Administrado pela 
Prefeitura do Campus de São Carlos, o CEFER-SC opera como o núcleo de atividades físicas e 
recreativas para os frequentadores da universidade na cidade. O espaço dispõe de ginásio, 
quadras externas, sala de musculação e piscina, tendo à mostra uma grade de atividades que 
inclui desde cursos de verão e treinos funcionais até modalidades coletivas, visando a promoção 
da saúde e a qualidade de vida no ambiente acadêmico.
 
\end{subsecao}

\begin{subsecao}{Onde beber?}
	
\quadrinhos{9}
Se vocês curtem entornar os canecos de vez em quando, então
agora devem estar pensando ``até que enfim vamos falar de algo que presta!'' --
só lembre-se pagar algumas doses para veteranes, já que são os responsáveis
por te dar essas dicas. Então, sem mais delongas, aqui vão
alguns lugares para se fazer isso à vontade:

{\bf FAU:} na vivência da FAU sempre vende cerveja. Para chegar, basta ir no 
primeiro andar à direita até o fim.

{\bf Física:} embora seja a Física, lá é um lugar gostoso para comer alguns
salgados na lanchonete e tomar algumas cervejas na atlética.

{\bf FFLCH:} vá até o prédio da História e Geografia e vá até o Aquário, fica 
logo à esquerda na entrada do vão, se estiver alguém lá dentro eles vendem cerveja.

{\bf FEA:} entrando na FEA, ande até o final, siga reto depois da biblioteca,
passe por um portãozinho, vire à esquerda e tem uma entrada antes do restaurante.
Lá é a vivência da FEA onde vendem as cervejas geladinhas.

%FIXME Qual é a situação atual do QiB?
% Perguntei pra alguns colegas, e não tive nenhuma resposta conclusiva.
% Atualização 2026/01: Acredito que não esteja mais rolando semanalmente
% por conta da reforma da prainha, falei disso e coloquei que ocorre regularmente
{\bf ECA:} famosa Quinta i Breja, adivinha que dia da semana isso acontece?
Acertou! Rola nas quintas-feiras na ECA a partir das 19h! Antigamente ocorria 
na prainha (a tal da vivência) mas, como está em reforma,
ocorre em diversos locais da ECA. Quem sabe a reforma finalmente não acaba esse ano???

{\bf Rei das batidas:} muito famoso não só por quem estuda na USP, o Rei,
como é carinhosamente chamado, fica fora da USP, saindo pelo P1. Vende
diversas batidas e, é claro, cerveja. Era mais popular, mas parece que já
perdeu a posição como o \textit{point} de encontro para o Beco (veja abaixo).

{\bf Bar do frango:} não se sabe qual é o verdadeiro nome desse bar, mas ele é
uma alternativa ao Rei, quando este se encontra muito lotado. Apesar do péssimo
atendimento e do aspecto horrível do lugar é bom para beber sem tumulto.
É frequentado principalmente no começo do ano. Se encontra atrás do Rei.

{\bf Beco da USP:} o famoso ``Beco''. Localizado perto da estação de metrô
Butantã, mais precisamente na Av. Valdemar Ferreira, 55. É um famoso boteco onde
muitos universitários costumam se reunir para descontrair, relaxar e desfrutar de
um menu de cervejas e porções mistas.

% FIXME: Coloquei o perfil do suquinho também, tranquilo né?
Não podemos esquecer das festas, que ocorrem em qualquer lugar da USP e quase
todas as sextas, e nelas há ainda outras misturas alcoólicas impossíveis. 
No instagram do Rolês da USP: \url{https://www.instagram.com/rolesdausp/} e do 
Suquinho0800: \url{https://www.instagram.com/suquinho0800/} eles compartilham as 
festas da semana geralmente (mas nem sempre todas).

Por fim, é nosso dever informar que o álcool é uma substância altamente viciante
e, quando bebida em excesso, pode trazer graves consequências à sua saúde, às
vezes à saúde de outra pessoa, à sua família e principalmente ao seu bolso.
Lembrem-se também que é proibida a venda de bebidas alcoólicas dentro da USP,
então não saiam da vivência dos respectivos lugares com a latinha na mão.

\end{subsecao}
\end{secao}
