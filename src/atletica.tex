\begin{secao}{A Atlética} %REFTIME - Quase tudo aqui é REFTIME...

\begin{subsecao}{O que é a AAAMat?}

É a Associação Atlética Acadêmica da Matemática - a entidade mais divertida do
mundo!!!! - que tem como objetivo trazer os melhores momentos da sua vida
universitária! A Atlética é formada por um grupo de IMEanes (gestão) que é
responsável por organizar atividades esportivas e eventos (festas, pizzadas,
premiações etc) para a comunidade IMEana.

\end{subsecao}
%%%%%%%%%%%%%%%%%%%%%%%%%%%%%%%%%%%%%%%%%%%%%%%%%%%%%%%%%%%%%%%%%%%%%%%%%%%%%%%%

Algumas das atividades da Atlética:

\begin{subsecao}{Atividades esportivas internas:}

Na AAAMat existem diretores de modalidade (DMs) que são pessoas responsáveis
pelos treinos e campeonatos dos seguintes esportes: futebol de campo, futsal,
basquete, vôlei, handebol, atletismo, natação, tênis de mesa, tênis de campo,
xadrez, baseball, softball, bridge, sinuca, truco, parkour, judô, karatê,
kendô, jiu-jitsu, rugby e e-sports (\textit{League of Legends}, \textit{Osu!},
\textit{Counter Strike: Global Offense}, \textit{Brawl Stars}, \textit{FIFA}
\textit{Rocket League}, \textit{Clash Royale}, \textit{Valorant},
\textit{Call of Duty: mobile}, \textit{Just Dance}, \textit{Hearthstone},
\textit{Pokemon UNITE}, \textit{Rainbow 6}, \textit{Super Smash Bros},
\textit{Teamfight Tactics}, \textit{Mario Kart},, \textit{Overwatch 2},
\textit{Legends of Runeterra}, \textit{Wild Rift} e outros estão sempre surgindo).
Além disso, contamos com a nossa amada BatIMEduca, bateria do IME juntamente
com a Pedago.

Os dias e horários dos treinos/jogos de cada uma dessas modalidades serão
sempre informados através do instagram e de um mural no vidro da salinha da atlética.
Vocês também podem entrar em contato pelas nossas redes sociais ou diretamente
na página da modalidade (todas estão na nossa bio do Instagram: @atleticaimeusp).

Com essa variedade de modalidades, não tem desculpa pro sedentarismo hein,
bixe? Se você não conhece nenhuma delas, a gente te apresenta e, se já conhece,
vem dar aquele “ooooi, sumido!!!” para aquele esporte que você largou por conta
da Fuvest! Ah, não esqueça de torcer por nosses atletas! Raça e coração, IME!

Além disso, a Atlética promove anualmente campeonatos internos daqueles jogos
que a gente passa hooooras fritando em casa. Já foram promovidos campeonatos de
\textit{Winning Eleven}, \textit{LoL}, \textit{Mario Kart}, \textit{Super Smash Bros}, 
\textit{Mario Tenis}, \textit{Guitar Hero} e 
sinuca.

Ideias e sugestões sobre novas modalidades, campeonatos, inters etc. são
sempre muito bem-vindas! Conversem com a gente!

\end{subsecao}
%%%%%%%%%%%%%%%%%%%%%%%%%%%%%%%%%%%%%%%%%%%%%%%%%%%%%%%%%%%%%%%%%%%%%%%%%%%%%%%%

\begin{subsecao}{Atividades esportivas externas:}

Além das modalidades da AAAMat, você pode participar das atividades do CEPEUSP,
o Centro de Práticas Esportivas da USP! 

O espaço oferece piscina (sim, piscina!!!), campos, quadras e salas incríveis.
Ainda organiza palestras, eventos e dicas sobre nutrição, saúde e bem-estar e 
possui cursos para a Terceira Idade, Comunidade USP e Externa. 

Os cursos vão de alongamento, caminhada e ioga até canoagem, capoeira, futsal, 
basquete, judô, karatê e outros. A duração média é de seis meses, possui custo 
significativo para a matrícula e as inscrições podem ser feitas direto pelo site. 

Para saber mais, visite o site do CEPEUSP: \url{https://www.cepe.usp.br} e acompanhe 
as datas. 

\end{subsecao}
%%%%%%%%%%%%%%%%%%%%%%%%%%%%%%%%%%%%%%%%%%%%%%%%%%%%%%%%%%%%%%%%%%%%%%%%%%%%%%%%

A AAAMat também representa o IME em diversos campeonatos universitários. São
eles:

\begin{subsecao}{BichUSP}

De nome intuitivo e charmoso, o BichUSP é um campeonato disputado entre as
faculdades da USP em que apenas bixes (VOCÊS!) participam. O campeonato
acontece logo nas primeiras semanas de aula, sempre aos finais de semana. Aqui,
vocês têm a chance de suar a camisa IMEana pela primeira vez e ver a torcida
indo ao delírio em cada jogada - ganhando ou perdendo, veteranes estarão
sempre vibrando por vocês!

``Mas, Atlética, eu não sei jogar nenhum desses esportes :('' - Não tem
problema, a gente te ensina! Teremos treinos especiais para que vocês conheçam
a modalidade, es DM’s, es técniques, a gente e outres bixes que te
acompanharão nesse momento único da graduação! O importante aqui é ter vontade
de participar e se divertir!

Fiquem atentos: a Atlética vai divulgar as datas do BichUSP em breve!

Se você acha que tem alergia a esportes, dá uma chance de a gente te mostrar
o contrário! São várias modalidades com várias dinâmicas diferentes, alguma
delas com certeza vai se encaixar no que você gosta! Se quiser só assistir
no começo e vir torcer com a gente, apareça nos jogos e nós vamos gritar
``VERMELHO E BRANCO ATÉ MORRER'' todos juntos!

%REFTIME
%Esse ano o BichUSP acontecerá nas seguintes datas:

%\begin{itemize}
  %\item Tênis todos os dias do BichUSP
  %\item 16 e 17/03/2019 - Basquete e Handebol.
  %\item 23 e 24/03/2019 - Natação, Atletismo, Tênis de Mesa e Xadrez.
  %\item 30/03 e 01/04/2019 - Futebol de campo e Rugby.
  %\item 06 e 07/04/2019 - Vôlei e Futsal.  
%\end{itemize}

Pra vocês se inspirarem, fizemos essa tabelinha que mostra quanto es bixes
brilharam em anos anteriores. Mal podemos esperar pra completar ela com as
conquistas que virão esse ano:

%REFTIME
\begin{center}
  \begin{tabular}{|c|c|}
    \hline
    Ano & Campeão\\
    \hline
    2005 & Basquete Masculino \\
    2005 & Tênis de Mesa Feminino \\
    2007 & Atletismo Masculino\\
    2009 & Atletismo\\
    2011 & Tênis de Campo Feminino\\
    2012 & Basquete Feminino\\
    2014 & Tênis de Mesa Masculino\\
    2014 & Futsal Masculino\\
    2015 & Futsal Masculino\\
    2016 & Vôlei Masculino\\
    2017 & Xadrez\\
    2017 & Rugby Misto\\
    2017 & Rugby Feminino (IME+EEFE)\\
    2018 & Xadrez\\
    2018 & Futebol de Campo Feminino\\
    2022 & Xadrez\\
    2023 & Brawl Stars\\
    2024 & BORA BIXES!!!!\\
    \hline
  \end{tabular}
\end{center}

\end{subsecao}
%%%%%%%%%%%%%%%%%%%%%%%%%%%%%%%%%%%%%%%%%%%%%%%%%%%%%%%%%%%%%%%%%%%%%%%%%%%%%%%%
\begin{subsecao}{Copa USP}

A Copa USP é o primeiro campeonato após o BichUSP, e existem duas séries (Azul:
1ª divisão e Laranja: 2ª divisão). São nesses jogos que colocamos em prática
tudo o que fizemos nos treinos semanais para brilharmos nos jogos da fase de
grupos e então seguirmos arrasando nos jogos mata-matas.

\end{subsecao}
%%%%%%%%%%%%%%%%%%%%%%%%%%%%%%%%%%%%%%%%%%%%%%%%%%%%%%%%%%%%%%%%%%%%%%%%%%%%%%%%
\begin{subsecao}{Jogos da Liga}

Acontece no segundo semestre (UFA, já passou a P1 de cálculo, vem Cálculo 2!!!
\sout{Ou não}) e nessa competição não existe separação por séries. Somente as
faculdades da USP jogam, as disputas são sorteadas para formarem grupos e
então apenas os melhores colocados seguem para a fase final. Essa é a
oportunidade perfeita para gritar um ``CHUPA, POLI!" na arquibancada.

\end{subsecao}
%%%%%%%%%%%%%%%%%%%%%%%%%%%%%%%%%%%%%%%%%%%%%%%%%%%%%%%%%%%%%%%%%%%%%%%%%%%%%%%%
\begin{subsecao}{NDU}

Esse campeonato acontece duas vezes ao ano, e várias faculdades de São Paulo
(tanto da USP quanto algumas não-USP) competem na cidade com rodadas durante
os fins de semana em busca dos melhores
resultados. Confira com ê DM da modalidade se o time está participando da
competição.

\end{subsecao}
%%%%%%%%%%%%%%%%%%%%%%%%%%%%%%%%%%%%%%%%%%%%%%%%%%%%%%%%%%%%%%%%%%%%%%%%%%%%%%%%
\begin{subsecao}{BIFE}

O BIFE É O MELHOR EVENTO ESPORTIVO DO MUNDO! As iniciais das quatro fundadoras
(Bio, IME, FAU e ECA) formam a sigla que dá nome a esses jogos universitários
que a gente tanto ama. Trata-se de um campeonato entre nove faculdades da USP:
a VET, Física, FFLCH, Química, Pedago e, é claro, as quatro fundadoras já 
citadas, às vezes contando também com algumas faculdades convidadas.

Funciona assim: em um determinado feriado, pessoas que jogam, torcem, festejam e
simpatizam se deslocam até alguma cidade do interior do Estado de SP. A cidade nos
dá um alojamento (leia-se: local para tomar um banho quentinho e descansar no
aconchego de sua barraca), alguns ginásios e um local para as festas. São
quatro dias, às vezes três, muito divertidos e engraçados, onde há rivalidade 
apenas dentro de quadra - porque fora é muito amor e integração! Além disso,
o BIFE organiza as três maiores festas da USP: Desmame, Engorda e Abate, as
quais antecedem a viagem para a cidade e servem para todo mundo se divertir e
se conhecer antes do evento principal!

Nosso histórico neste Inter é de parar o trânsito! Olhem só:

%REFTIME
\begin{center}
  \begin{tabular}{|c|c|c|}
   \hline
   Ano & Cidade & Campeão\\
   \hline
   1999 & Jacareí & IME\\
   2000 & Não Houve & - \\
   2001 & Serra Negra & IME\\
   2002 & Socorro & ECA\\
   2003 & São Sebastião & IME\\
   2004 & Cruzeiro & FFLCH\\
   2005 & Jacareí & FFLCH\\
   2006 & Lorena & IME\\
   2007 & Piedade & IME\\
   2008 & Itapeva & IME\\
   2009 & Cruzeiro & IME\\
   2010 & Barra Bonita & IME\\
   2011 & Casa Branca & IME\\
   2012 & Barra Bonita & IME\\
   2013 & Sumaré & ECA\\
   2014 & Cidade/Araraquara & FFLCH\\
   2015 & Taquaritinga & FFLCH\\
   2016 & Registro & FFLCH\\
   2017 & Avaré & FFLCH\\
   2018 & Casa Branca & FAU\\
   2019 & Franca & FFLCH\\
   2020 & On-line & AAAGW\\
   2021 & On-line & IME\\
   2022 & Itapeva & ICBIÓ\\
   2023 & Casa Branca & ICBIÓ\\
   2024 & VAMO IME!! & \\
   \hline
  \end{tabular}
\end{center}

%REFTIME
Em 2020 e 2021, o BIFE se reinventou trazendo o inter de maneira on-line, através do
E-Bife, em que rolaram jogos dos e-sports e também algumas festas. Nas edições de 2022 e 2023 nós ficamos em  4º!!!!! Nossos times contam com vocês para conquistar
mais um título em 2025! VAMO QUE VAMO, GALERA!!!


\end{subsecao}
%%%%%%%%%%%%%%%%%%%%%%%%%%%%%%%%%%%%%%%%%%%%%%%%%%%%%%%%%%%%%%%%%%%%%%%%%%%%%%%%

\begin{subsecao}{Títulos}

Fruto de muito treino, empenho, suor, torcida e amor pelo IME-USP, reunimos
abaixo algumas de nossas conquistas:

%REFTIME
\begin{center}
  \begin{tabular}{|c|c|c|c|}
    \hline
    Ano & Campeonato & Modalidade & Colocação\\
    \hline
    2009 & Intercalouros  & Atletismo       & 1º\\
    2011 & LUPAA          & Atletismo       & 1º\\
    2011 & BOBPAI         & Baseball        & 1º\\
    2016 & BOBPAI         & Baseball        & 2º\\
    2016 & Liga Paulista  & Baseball        & 3º\\
    2017 & Copa USP       & Baseball Fem.   & 2º\\
    2016 & Wakaba         & Softball        & 2º\\
    2016 & Softparty      & Softball        & 1º\\
    2023 & Copa USP       & Softball        & 2º\\
    2011 & Jogos da Liga  & Basquete Fem.   & 1º\\
    2012 & Copa Camp      & Basquete Fem.   & 2º\\
    2012 & Copa USP       & Basquete Fem.   & 3º\\
    2019 & BIFE           & Basquete Fem.   & 2º\\
    2018 & BIFE           & Basquete Masc.  & 1º\\
    2018 & Jogos da Liga  & Basquete Masc.  & 2º\\
    2018 & Jogos da Liga Série Prata & Basquete Masc. & 1º\\
    2019 & BIFE           & Basquete Masc.  & 1º\\
    2022 & BIFE           & Basquete Masc.  & 2º\\
    2023 & NDU            & Basquete Masc.  & 3º\\
    2023 & Copa USP Série Laranja & Basquete Masc. & 1º\\
    2020 & E-BIFE         & Brawl Stars     & 1º\\
    2021 & E-BIFE         & Brawl Stars     & 1º\\
    2020 & E-BIFE         & Clash Royale    & 1º\\
    2021 & E-BIFE         & Clash Royale    & 1º\\
    2021 & E-BIFE         & CSGO            & 3º\\
    2018 & BIFE           & E-Sports        & 1º\\
    2019 & BIFE           & E-Sports        & 2º\\
    2020 & E-BIFE         & E-Sports        & 2º\\
    2021 & E-BIFE         & E-Sports        & 1º\\
    2022 & BIFE           & E-Sports        & 1º\\
    2023 & BIFE           & E-Sports Absoluto  & 1º\\
    2023 & BIFE           & E-Sports Inclusivo & 1º\\
    2015 & Integramix     & Futebol Campo   & 1º\\
    2017 & Copa USP       & Futebol Campo M & 1º\\
    2018 & BIFE           & Futebol Campo F & 2º\\
    2014 & Interfarofa    & Futsal Fem.     & 1º\\
    2015 & NDU            & Futsal Fem.     & 2º\\
    2017 & Copa USP       & Futsal Fem.     & 1º\\
    2017 & Jogos da Liga  & Futsal Fem.     & 2º\\
    2017 & IMEACHECA      & Futsal Fem.     & 1º\\
    2022 & Interatléticas & Futsal Fem.     & 2º\\
    2011 & Copa USP       & Futsal Masc.    & 1º\\
    2012 & Copa Camp      & Futsal Masc.    & 1º\\
    2013 & NDU            & Futsal Masc.    & 1º\\
    2013 & Jogos da Liga  & Futsal Masc.    & 2º\\
    2015 & Integramix     & Futsal Masc.    & 1º\\
    2015 & NDU            & Futsal Masc.    & 2º\\
    2023 & Jogos da Liga Série Prata & Futsal Masc. & 2º\\
    2023 & Interatléticas & Futsal Masc.    & 1º\\
    2023 & BIFE           & Futsal Masc.    & 1º\\
    2015 & Camp. G-4      & Handebol Fem.   & 1º\\
    \hline
         \end{tabular}
\end{center}
\begin{center}
\begin{tabular}{|c|c|c|c|}
  \hline
  Ano & Campeonato & Modalidade & Colocação\\
  \hline
    2015 & Integramix     & Handebol Fem.   & 1º\\
    2016 & CUPA           & Handebol Fem.   & 2º\\
    2016 & Interfarofa    & Handebol Fem    & 1º\\
    2018 & Copa USP       & Handebol Fem.   & 1º\\
    2006 & Jogos da Liga  & Handebol Masc.  & 2º\\
    2009 & Copa USP       & Handebol Masc.  & 1º\\
    2012 & Copa USP       & Handebol Masc.  & 2º\\
    2016 & IMEACHECA      & Handebol Masc.  & 2º\\
    2018 & Jogos da Liga  & Handebol Masc.  & 2º\\
    2022 & BIFE           & Handebol Masc.  & 1º\\
    2023 & Jogos da Liga Série Prata & Handebol Masc. & 2º\\
    2023 & Copa USP Série Laranja & Handebol Masc.  & 1º\\
    2017 & TUES           & Hearthstone     & 2º\\
    2020 & E-BIFE         & Hearthstone     & 1º\\
    2017 & Copa USP       & Jiu-jitsu       & 2º\\
    2022 & BIFE           & Judô Masc.      & 2º\\
    2023 & BIFE           & Judô Masc.      & 2º\\
    2021 & E-BIFE         & LOL             & 2º\\
    2018 & BIFE           & Natação Masc.   & 2º\\
    2018 & Jogos da Liga  & Natação Masc.   & 2º\\
    2019 & BIFE           & Natação Masc.   & 3º\\
    2023 & BIFE           & Natação Masc.   & 1º\\
    2021 & E-BIFE         & Rocket League   & 2º\\
    2018 & BIFE           & Rugby Masc.     & 2º\\
    2022 & BIFE           & Sinuca          & 1º\\
    2023 & BIFE           & Sinuca          & 1º\\
    2017 & Jogos da Liga  & Tênis Campo M   & 2º\\
    2021 & E-BIFE         & Valorant        & 2º\\
    2016 & IMEACHECA      & Vôlei Fem.      & 1º\\
    2016 & Gran Prix USP  & Vôlei Fem.      & 3º\\
    2017 & Gran Prix USP  & Vôlei Fem.      & 3º\\
    2015 & Camp. G-4      & Vôlei Masc.     & 2º\\
    2016 & Copa USP       & Vôlei Masc.     & 1º\\
    2017 & Copa USP       & Vôlei Masc.     & 2º\\
    2017 & NDU            & Vôlei Masc.     & 1º\\
    2018 & Copa USP       & Vôlei Masc.     & 1º\\
    2018 & BIFE           & Vôlei Masc.     & 2º\\
    2018 & NDU            & Vôlei Masc.     & 2º\\
    2023 & Jogos da Liga Série Prata & Vôlei Masc. & 2º\\
    2017 & NDU            & Xadrez          & 1º\\
    2017 & Copa USP       & Xadrez          & 3º\\
    2017 & Jogos da Liga  & Xadrez          & 2º\\
    2020 & E-BIFE         & Xadrez          & 1º\\
    2021 & E-BIFE         & Xadrez          & 1º\\
    2022 & BIFE           & Xadrez          & 3º\\
    2023 & NDU            & Xadrez          & 3º\\
    2023 & Copa USP       & Xadrez          & 1º\\
    2023 & BIFE           & Xadrez          & 1º\\
    \hline
  \end{tabular}
\end{center}

\end{subsecao}
%%%%%%%%%%%%%%%%%%%%%%%%%%%%%%%%%%%%%%%%%%%%%%%%%%%%%%%%%%%%%%%%%%%%%%%%%%%%%%%%
\begin{subsecao}{Outras atividades}

\begin{subsubsecao}{Vendas}

A Atlética também quer te ajudar a vestir o vermelho e branco (que, a essa
altura, já corre em suas veias! :D) e traz pra você diversos produtos
personalizados, tais como: canecas, talabartes, corta-ventos,
samba-canção, camisetas e agasalhos do
IME, pra você sair por aí esbanjando seu amor pelo IME-USP $<$3

Vocês podem comprar esses produtos na salinha da Atlética, ou nos nossos
stands de venda.

\end{subsubsecao}
%%%%%%%%%%%%%%%%%%%%%%%%%%%%%%%%%%%%%%%%%%%%%%%%%%%%%%%%%%%%%%%%%%%%%%%%%%%%%%%%
\begin{subsubsecao}{Festas}

%REFTIME
A Atlética e o CAMat já promoveram muitas festas e happy hours. Atualmente,
promovemos a I Will SurvIME, a JunIME, a Melhores do Ano e, além disso, 
auxiliamos as festas pré-BIFE (Desmame, Engorda e Abate). Todas imperdíveis!
Esperamos vocês!

\end{subsubsecao}
%%%%%%%%%%%%%%%%%%%%%%%%%%%%%%%%%%%%%%%%%%%%%%%%%%%%%%%%%%%%%%%%%%%%%%%%%%%%%%%%
\begin{subsubsecao}{Como falar com a Atlética?}

A salinha da AAAMat é a B-18. Ela fica dentro da vivência e é pequena, mas
sempre cabe mais um! Sempre que precisarem conversar com a Atlética, vocês
podem ir até lá e falar com qualquer membro da gestão. Além disso, também temos
outros meios de contato, tais como o e-mail: aaamat.presidente@gmail.com  e as
nossas redes sociais (principalmente o Instagram)

Para ficarem sabendo tudo que acontece na atlética basta:

\begin{itemize}
  \item Acompanhar o site da atlética: \url{https://atletica.ime.usp.br}
  \item Curtir nossa página no Facebook: \url{https://fb.com/aaamat.ime}
  \item Seguir a gente no Instagram: @atleticaimeusp
\end{itemize}

%REFTIME
Talvez vocês notem que existem outros dois instas antigos da AAAMat
(@aaamat\_imeusp e @atleticaime), ambos já não são mais utilizados
e se algum bixe souber como derrubar essas contas, fale para a atlética, 
pois a gestão será eternamente agradecida pela sua ajuda! 

A Atlética inicia 2024 esperando vocês, bixes querides, para que juntos
possamos trazer muitos títulos e troféus para casa!! É importantíssimo que
vocês saibam que estamos abertos para qualquer tipo de crítica, dúvida, ideia
ou sugestão.

Vocês são SEMPRE muito bem-vindes em nossa sala, atividades, times e eventos!
\end{subsubsecao}
\end{subsecao}
\end{secao}
\pagebreak
