\begin{secao}{USP Virtual}

\begin{subsecao}{E-disciplinas e E-mail USP}

A USP tem, há alguns anos, um ambiente virtual de aprendizagem para dar apoio
às disciplinas. É o Moodle da nossa universidade, oficialmente conhecido como
o e-Disciplinas (é comum chamarmos dos dois jeitos e há alguns professores que
se referem a ele como ``Stoa'', que é um nome bem antigo!). Você pode acessá-lo em
\url{https://edisciplinas.usp.br/}, fazendo login com o e-mail USP e a senha única.

Na página inicial, você provavelmente encontrará links para as páginas da maioria
das disciplinas que vai cursar (se não todas). Muitos professores não utilizam o 
e-Disciplinas como apoio aos seus cursos, às vezes fazem isso nos seus próprios 
sites institucionais ou simplesmente não usam a internet para nada. 

Já falamos outras vezes aqui neste guia, mas como esse tópico é sobre a USP Virtual,
não custa nada relembrar: acesse seu e-mail USP todos os dias. Os professores que não 
utilizam o e-Disciplinas ou um site institucional, costumam manter tudo em dia pelo 
e-mail, sejam datas de provas, pdf de listas, lembretes. Logo, seu e-mail USP será mais 
um companheiro nesses anos que virão por aí! Vale também acessar o drive compartilhado 
da matéria (caso tenha!), ali pode ser que sejam postados as listas, resoluções dos 
professores e afins. Peçam para os veteranes falarem mais um pouco de algumas 
(das \textbf{muitas}) vantagens de se ter um e-mail USP e aproveitem tudo aquilo que 
vocês agora têm direito!

\end{subsecao}

\begin{subsecao}{Grupos no Whatsapp}

Com o intuito de organizar as atividades referentes a uma disciplina e ser uma
maneira prática e efetiva de passar recados, costumamos utilizar grupos do WhatsApp
para cada uma das matérias. Geralmente alguns cursos organizam e disponibilizam uma planilha 
com os links de convite para cada um dos grupos. Caso os links ainda não tenham sido criados,
vale a pena juntar seus amigos para criar os grupos no começo do semestre e espalhar
para os outros alunos.

Dentre as principais funções do grupo do WhatsApp estão:

\begin{enumerate}
\item Compartilhar os materiais e listas disponibilizados pelo docente da disciplina;
\item Passar recados importantes, sejam datas para entrega de trabalho, provas etc.;
\item Tirar dúvidas sobre a matéria com os coleguinhas.
\end{enumerate}

Pra ficar por dentro de tudo o que acontece na matéria e também interagir com seus colegas
de turma, não se esqueça de procurar o link do grupo de whats da matéria que você está fazendo
e faça parte do grupo também!! :)

\end{subsecao}


\end{secao}
