\begin{secao}{Hospital Universitário}
   \begin{quote}\emph{O HU USP é o hospital de ensino de excelência utilizado
pelos Cursos de Atenção à Saúde da USP.  O hospital privilegia as pesquisas
relacionadas aos problemas de saúde  mais comuns da população brasileira. O
atendimento é regionalizado para o bairro do Butantã, sempre com enfoque no
ensino e pesquisa''}- trecho obtido da página do HU (\url{http://www.hu.usp.br/})
\textit{alguns} anos atrás.
   \end{quote}

Isso significa que os estudantes de Medicina, Ciências Farmacêuticas,
Odontologia, Saúde Pública, da Escola de Enfermagem e do Instituto de
Psicologia, mantendo contato direto também com os Institutos de Ciências
Biomédicas, de Biologia, de Química, a Faculdade de Arquitetura e Urbanismo,
a Escola Politécnica e a Escola de Comunicações e Artes precisam de cobaias para
suas atividades/experiências. O \sout{USPital} HU é um santo lugar que recebe alguns 
fracos de espírito que bebem demais e ficam incapacitados de fazer qualquer 
atividade fisiológica. 

Qualquer pessoa pode fazer seu cadastro no Hospital Universitário. Para isso, é 
necessário comparecer pessoalmente ao hospital portando os seguintes documentos:

\begin{itemize}
   \item a carteirinha USP;
   \item um documento oficial com foto (RG, CNH...);
   \item o CPF;
   \item o CNS (Cartão Nacional de Saúde, do SUS).
\end{itemize}

Estudantes da USP podem realizar esse cadastro a partir de um formulário 
disponibilizado no site do HU: (\url{http://www.hu.usp.br/}). É nesse mesmo site 
que você poderá agendar consultas e exames, cuide da sua saúde bixe!

Assim vocês possuirão alguns privilégios no atendimento, em situações de
emergência vocês são atendidos rapidamente (algo entre 600 minutos, como
diz na senha de espera). %e não precisam soletrar o nome da mãe enquanto
%estiverem desmaiados.

\begin{description}
\item [Como chegar:] Pegar 8012-10 ou 8022-10 sentido cidade universitaria
\item [Endereço:] Av. Prof. Lineu Prestes, 2565 - Butantã, São Paulo - SP, 05508-000
\item [Telefone:] (11) 3091-9200
\end{description}

Mas nem tudo são flores, HU não vai te atender para tudo, para muitos casos 
(99,9% das vezes) você vai ter que usar os serviços de uma UBS (Unidade Básica de Saúde),
caso seja de fora da cidade de São Paulo, confira no mapa qual UBS atende onde você mora
(\url{https://www.google.com/maps/d/viewer?mid=1-cDHamkJnuwbxrOEZVK7MLb7PoF2n88_&ll=-23.68533875879342%2C-46.521722701763906&z=10})
leve com você documento com foto, comprovante de residencia e o cartãozinho do SUS.

Feito o registro, a primeira coisa que você ira fazer, sera agendar um clinico geral para fazer um exame dermatologico,
com o laudo em mãos, você ira leva-lo ao CEPE para assim poder usar as piscinas de lá!

\end{secao}
