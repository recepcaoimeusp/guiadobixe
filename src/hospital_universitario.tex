\begin{secao}{Hospital Universitário}
   \begin{quote}\emph{O HU USP é o hospital de ensino de excelência utilizado
pelos Cursos de Atenção à Saúde da USP.  O hospital privilegia as pesquisas
relacionadas aos problemas de saúde  mais comuns da população brasileira. O
atendimento é regionalizado para o bairro do Butantã, sempre com enfoque no
ensino e pesquisa''}- trecho obtido da página do HU (\url{http://www.hu.usp.br/})
\textit{alguns} anos atrás.
   \end{quote}

Isso significa que os estudantes de Medicina, Ciências Farmacêuticas,
Odontologia, Saúde Pública, da Escola de Enfermagem e do Instituto de
Psicologia, mantendo contato direto também com os Institutos de Ciências
Biomédicas, de Biologia, de Química, a Faculdade de Arquitetura e Urbanismo,
a Escola Politécnica e a Escola de Comunicações e Artes precisam de cobaias para
suas atividades/experiências. O \sout{USPital} HU é um santo lugar que recebe alguns 
fracos de espírito que bebem demais e ficam incapacitados de fazer qualquer 
atividade fisiológica. 

O cadastro HU é feito para integrantes da comunidade USP (docentes, funcionários,
alunos de graduação ou Pós-Graduação) e dependentes cadastrados. Para isso, é 
necessário comparecer pessoalmente ao hospital portando os seguintes documentos (físicos ou digitais):

\begin{itemize}
   \item a carteirinha USP;
   \item um documento oficial com foto (RG, CNH...);
   \item o CPF;
   \item o CNS (Cartão Nacional de Saúde, do SUS).
\end{itemize}

Os alunos da USP podem realizar esse cadastro a partir de um formulário 
disponibilizado no site do HU: (\url{https://www.hu.usp.br/cadastro-hu/}). É nesse mesmo site 
que você poderá agendar consultas e exames. Cuide da sua saúde, bixe!

Assim vocês possuirão alguns privilégios no atendimento, em situações de
emergência vocês são atendidos rapidamente (algo entre 600 minutos, como
diz na senha de espera). %e não precisam soletrar o nome da mãe enquanto
%estiverem desmaiados.

Para quem não é da comunidade USP, o agendamento é realizado pela Unidade
Básica de Saúde (UBS) através do Sistema CROSS.

\begin{description}
\item [Endereço:] Av. Prof. Lineu Prestes, 2565 - Butantã, São Paulo - SP, 05508-000
\item [Telefone:] (11) 3091-9200
\end{description}

\begin{subsecao}{Outros hospitais da USP:}

{\bf Hospital das Clínicas - São Paulo:} Conhecido como o maior complexo hospitalar 
da América Latina, o Hospital das Clínicas da FMUSP (HCFMUSP) em São Paulo funciona 
como um centro de referência mundial em ensino, pesquisa e assistência de alta complexidade. 
O complexo tem à mostra uma estrutura gigantesca composta por diversos institutos 
especializados (como o InCor e o ICESP), oferecendo atendimento multidisciplinar pelo SUS e 
servindo como campo de prática para a formação de profissionais de saúde da USP.

{\bf Hospital das Clínicas - Ribeirão Preto:} Situado no interior paulista, o Hospital das 
Clínicas de Ribeirão Preto (HCFMRP) funciona como hospital-escola vinculado à FMRP-USP e é 
a principal referência em saúde pública para milhões de habitantes da região. A instituição 
oferece atendimento de nível terciário e quaternário (alta complexidade), tendo à mostra 
unidades de emergência, ambulatoriais e cirúrgicas que combinam assistência humanizada com 
o desenvolvimento de pesquisas clínicas de ponta.

{\bf Hospital de Reabilitação de Anomalias Craniofaciais - Bauru (HRAC/Centrinho):} Localizado 
em Bauru, o Hospital de Reabilitação de Anomalias Craniofaciais (HRAC/Centrinho) funciona 
como um centro de excelência pioneiro no tratamento de fendas labiopalatinas e deficiências 
auditivas. Reconhecido pela OMS, o hospital tem à mostra um modelo de atendimento interdisciplinar 
integral, oferecendo desde cirurgias reparadoras até terapias de fonoaudiologia e suporte 
psicológico, transformando a vida de pacientes de todo o Brasil.

{\bf Hospital Veterinário - São Paulo (HOVET):} Dentro da Cidade Universitária em São Paulo, 
o Hospital Veterinário (HOVET) da FMVZ-USP funciona como o maior centro de residência e 
aprimoramento veterinário da América Latina. Aberto à comunidade, o hospital tem à mostra 
serviços clínicos e cirúrgicos para pequenos e grandes animais, além de animais silvestres, 
contando com laboratórios de diagnóstico avançado e diversas especialidades médicas para o 
tratamento de pets e animais de produção.

{\bf Hospital Veterinário - Pirassununga (HV-FZEA):} Situado no campus de Pirassununga, 
o Hospital Veterinário da FZEA (HV-FZEA) funciona como suporte prático para o curso de 
Medicina Veterinária, com forte vocação para a área de grandes animais e produção animal, 
dado o perfil do campus. O local oferece atendimento clínico e cirúrgico para a comunidade 
regional, tendo à mostra infraestrutura para o tratamento de equinos, bovinos e pequenos 
animais, integrando a prestação de serviços às atividades de ensino e pesquisa em zootecnia 
e veterinária.

\end{subsecao}

\end{secao}
