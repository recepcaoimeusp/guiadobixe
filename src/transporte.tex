\begin{secao}{Tudo Que Vai Volta (até bixes)}

\begin{subsecao}{Ônibus}

Se vocês não têm como ir nem como voltar, temos algumas dicas:

\begin{enumerate}
\vspace{-15pt}
  \item Trabalhem muito para comprar um carro, trabalhem mais para pagar a
  gasolina, venham para a USP de carro e, obrigatoriamente, deem carona a um
  veterane;

  \item Peçam a uma pessoa amiga para trazê-los e buscá-los durante seus
  longos anos de IME-USP;

  \item Conheçam alguém que, por sorte, more perto da sua casa, estude na USP,
  tenha o mesmo horário que você, seja legal e tenha carro.
  Traduzindo: s-o-n-h-e;

  \item Estiquem o dedão e esperem, esperem, esperem, esperem... a boa vontade
  de alguém para dar carona;

  \item Mudem-se para uma casa perto da USP;

  \item Peguem uma bicicleta e pedalem.

\end{enumerate}
\vspace{-15pt}
OK, vocês decidiram ser bixes normais e pegar ônibus! Nesse caso, vocês
provavelmente estão em uma das seguintes opções:
\vspace{-15pt}
\begin{enumerate}
  \item Vocês vêm de metrô pra USP;
  \item Vocês são super sortudos e acharam um ônibus que vai de casa pra USP.
\end{enumerate}
\vspace{-15pt}
Se vocês vão usar ônibus para ir ou voltar da USP, vocês precisam conhecer
os quatro pontos de ônibus perto do IME: FAU I, Oceanografia, FAU II e FEA.
Até 2019, esses quatro pontos tinham placas com os nomes escritos, mas
agora foram reformados e não têm mais as placas. Mesmo assim, a localização
deles não poderia ser mais intuitiva, veja só:

Tanto o ponto FAU I quanto o ponto da Oceanografia se localizam na Rua do
Matão -- FAU I é o ponto que fica mais próximo da FAU, e o da Oceanografia é o
que fica do mesmo lado da calçada que o IO (pronuncia-se i-ó), Instituto de
Oceanografia; os pontos FAU II e FEA ficam na Av. Prof. Luciano Gualberto (que a
partir de agora será chamada de Rua dos Bancos, para todo e todo o sempre),
e, como vocês já devem estar imaginando, são na frente da FAU e da FEA,
respectivamente. São muitos nomes para lembrar? Fiquem tranquilos: mais
cedo ou mais tarde vocês vão conhecer esses lugares e não vão nem se preocupar
mais com os nomes.

Como vocês viram ali em cima, nós damos apelidos também intuitivos para as
principais avenidas do campus:
\begin{itemize}
	\item Av. Prof. Luciano Gualberto = Rua dos Bancos;
	\item Av. Prof. Lineu Prestes = Rua do HU;
	\item Av. Prof. Mello Moraes = Rua da Raia.
\end{itemize}

Aqui está a lista de ônibus que passam em cada um dos pontos. Para mais detalhes
sobre cada linha, vocês podem usar o site da SPTrans ou o Google Maps.

\begin{subsubsecao}{Linhas municipais}

{\bf Ponto FAU I}

\begin{center}
	\begin{tabular}{|c|c|c|}
      \hline
	  Letreiro & Cor & Interligações (em ordem)\\
	  \hline
	  8012-10 - Metrô Butantã* & Laranja & Metrô: L4\\
   	8082-10 - Metrô Butantã* & Laranja & Metrô: L4\\
      \hline
	\end{tabular}
\end{center}

{\bf Ponto da Oceanografia}

\begin{center}
	\begin{tabular}{|c|c|c|}
      \hline
	  Letreiro & Cor & Interligações (em ordem)\\
	  \hline
	  8022-10 - Metrô Butantã* & Laranja & Metrô: L4\\
   	  8082-10 - Cid. Universitária* & Laranja & Metrô: L4\\
      \hline
	\end{tabular}
\end{center}

{\bf Ponto FAU II}

(Esses os levam pra fora da USP)
\begin{center}
	\begin{tabular}{|c|c|c|}
      \hline
	  Letreiro & Cor & Interligações (em ordem)\\
	  \hline
	  177H-10 - Metrô Santana & Azul & Metrô: L4, L2, L3 e L1\\
	  7181-10 - Term. Princ. Isabel & Laranja & CPTM: L9\\
	  7411-10 - Praça da Sé & Laranja & Metrô: L4, L2, L3 e L1\\
	  7725-10 - Rio Pequeno & Laranja & - \\
	  809U-10 - Metrô Barra Funda & Laranja & L2 \\
	  8022-10 - Term. USP P3* & Laranja & -\\
   	8086-10 - Pinheiros & Laranja & CPTM: L9, Metrô: L4 \\
   	8084-10 - Metrô Butantã* & Laranja & Metrô: L4\\
    8085-10 - Circular USP / P3* & Laranja & -\\
      \hline
	\end{tabular}
\end{center}

Esses dois ônibus passam no ponto, mas estão CHEGANDO na USP, preste 
atenção!

\begin{center}
	\begin{tabular}{|c|c|}
	  \hline
	  Letreiro & Cor\\
	  \hline
	  701U-10 - Butantã-USP & Azul\\
	  702U-10 - Butantã-USP & Laranja\\
	  \hline
	\end{tabular}
\end{center}

{\bf Ponto FEA}

(Esses os levam pra fora da USP)
\begin{center}
	\begin{tabular}{|c|c|c|}
      \hline
	  Letreiro & Cor & Interligações (em ordem)\\
	  \hline
	  701U-10 - Metrô Santana & Azul & Metrô: L4, L2, L3 e L1\\
	  702U-10 - Term. Pq. D. Pedro II & Laranja & Metrô: L4, L2 e L3\\
	  7725-10 - Terminal Lapa & Laranja & CPTM: L8\\
	  8012-10 - Term. USP P3* & Laranja & -\\
	  8084-10 - Cid. Universitária* & Laranja & Metrô: L4\\
    8085-10 - Circular USP / P3* & Laranja & -\\

      \hline
	\end{tabular}
\end{center}

Novamente, esses quatro ônibus passam no ponto, mas estão CHEGANDO na USP!
\begin{center}
	\begin{tabular}{|c|c|}
	  \hline
	  Letreiro & Cor\\
	  \hline
	  177H-10 - Butantã-USP & Azul\\
	  7181-10 - Cidade Universitária & Laranja\\
	  7411-10 - Cidade Universitária & Laranja\\
	  809U-10 - Cidade Universitária & Laranja\\
	  \hline
	\end{tabular}
\end{center}

As linhas marcadas com um * são as linhas circulares da SPTrans.
Segue abaixo suas descrições!

\end{subsubsecao}

\begin{subsubsecao}{Circulares}

Também conhecido como ``circulenda'' ou ``secular'' (aos sábados e domingos,
``milenar''), é o meio de transporte mais barato dentro da USP. Foi criado para
os USPianos se locomoverem dentro do Campus, mas em muitas vezes é melhor andar
do que ficar esperando. Existem 6 itinerários distintos: 2 deles têm trajetos
aproximadamente reversos e que juntos cobrem todo o Campus; um outro é uma
``versão expressa" que agrega pontos da Av. Prof Luciano Gualberto (Rua dos 
Bancos), onde os outros itinerários não passam (isso inclui o IME, como vocês 
devem ter visto na tabela do ponto da FEA aí em cima); outra linha é exclusivamente 
interna, ou seja, o ônibus circula do P3 ao P1 sem sair do Campus; já os 2 itinerários 
que restam funcionam apenas aos finais de semana e durante a madrugada. Isso já 
vai ser explicado melhor. Fiquem atentos para não se perderem, hein?

Essas linhas funcionam dentro da USP e nós, alunos, não pagamos, pois elas aceitam o 
bilhete USP (BUSP -- Retire logo o seu!!).

Em 2024 foram implantadas as linhas 8082/10 (circular 1), 8083/10 (circular 2), 
8084/10 (circular 3) e 8085/10 (circular interno) substituindo as linhas 8012/10 e 
8022/10 durante os dias da semana -- porém, as linhas 8012/10 e 8022/10 ainda funcionam 
durante o período noturno (entre 00:17h e 03:15h) e aos finais de semana. Essas linhas 
funcionam dentro da USP, e nós alunos não pagamos, pois elas aceitam o bilhete USP 
(BUSP -- Retire logo o seu!!).

{\bf Circulares durante os fins de semana}

Durante os fins de semana e madrugadas há uma mudança no funcionamento dos 
circulares, as linhas lançadas em 2024 saem de cena e dão lugar as
antigas linhas, implantadas em 2012, 8012-10 e 8022-10, além disso o
intervalo entre os ônibus aumenta significativamente.

%REFTIME
Em 2021 houve uma grande mudança no trajeto dos circulares, porque agora eles
têm um ponto final no Portão 3 (e tecnicamente não são mais circulares, será que
o nome vai ficar?). Como essa mudança ocorreu durante a pandemia, até mesmo
veteranes podem ser surpreendidos, então prestem atenção nos mapas!
=======
8084/10 (circular 3) e 8085/10 (circular interno), substituindo as linhas 8012/10 e 
8022/10 durante os dias da semana -- mas fiquem atentos! As linhas 8012/10 e 8022/10 
ainda funcionam durante o período noturno (entre 00:17h e 03:15h) e aos finais de semana. 

Os ônibus 8082/10 (circular 1) e 8083/10 (circular 2) começam seus trajetos no 
Terminal Butantã com ponto final no Portão 3. O ônibus 8084/10 (circular 3) também 
parte do Terminal Butantã, mas ele volta para o próprio Terminal Butantã no fim do trajeto. 
Já o circular 8085/10 (circular interno) parte do Portão 3, e volta para ele no fim 
do trajeto, não passando pelo Butantã.

Para acompanharem as rotas dos circulares em tempo real, bem como ver pontos
importantes da faculdade, podem usar os aplicativos ``Guia USP", "Moovit" ou o
Google Maps.

Para os bixes que não gostam de ler (preferem imagens) colocamos o
mapa das quatro linhas nas próximas páginas.


\mapa{8082.jpg}
\mapa{8083.jpg}
\mapa{8084.jpg}
\mapa{8085.jpg}
\mapa{8012_ida.png}
\mapa{8012_volta.png}
\mapa{8022_ida.png}
\mapa{8022_volta.png}

\end{subsubsecao}

\begin{subsubsecao}{Como ir e vir do IME pelo metrô Butantã}

Como sabemos que vocês, bixes, chegam muito perdidos, e muitos pularam todas
essas informações sobre os pontos de ônibus e por isso podem não saber como
fazer seu trajeto, resolvemos colocar aqui um dos trajetos mais comuns que boa
parte de vocês vão usar.

{\bf Circulares durante dias de semana}

Quando estiver fazendo o trajeto entre Terminal Butantã e IME priorize o circular 3
(8084-10), pois ele é bem mais rápido. O circular 1 (8082-10) também é uma opção, mas
ele dá uma volta maior pelo campus.

O circular 3 (8084-10) tem as seguintes paradas até o IME: fora da USP, ele
passa pela parada Vicente Rogrigues e pela Afrânio Peixoto; dentro do campus,
passa pela Educação, Biblioteca Brasiliana, Praça dos Bancos, FEA, Poli-Biênio, 
Poli-Eletrotécnica e FAU II, o ponto mais próximo ao IME. A parada mais rápida 
(e onde a maioria dos imeanos desce) é a da FEA, que fica do outro lado da 
rua -- aí é só atravessar que você chega ao IME.

Quando estiver indo para o metrô Butantã, pegue o circular 3 (8084-10) no ponto 
da FAU II. O ponto final dele (e do 8082-10) é no próprio metrô, então só precisa 
descer quando ele chegar ao fim do trajeto.

Se você optar por utilizar o circular 1 (8082-10), saiba que ele faz o seguinte 
trajeto até chegar ao IME: Educação Física II, CPTM II, Raia Olímpica, Psicologia II, 
Portão 2, Terminal USP, IPT, Prefeitura I, Física e Oceanográfico, onde você deve 
descer. Na volta, para ir para o metrô Butantã, pegue o circular no ponto da FAU I, 
que fica na frente do ponto da Oceanografia.

{\bf Circulares durante finais de semana}

Durante os fins de semana e madrugadas, para chegar no IME a partir do 
metrô Butantã, vocês devem pegar o 8012-10. Se vocês pegarem o 8022-10, 
em algum momento vocês vão chegar no IME, mas o 8012-10 é muito mais rápido.

O 8012-10 vai fazer o seguinte trajeto: Faculdade de Educação,
Praça do Relógio Solar, Biblioteca Brasiliana, Praça dos Bancos e
FEA. A partir daí, deverão descer no ponto da FEA.

Quando estiverem indo embora para o metrô, vocês devem pegar o 8022-10
no ponto da Oceanografia. CUIDADO: não peguem o 8012-10 aqui
se estiverem querendo ir para o metrô Butantã!!! Lembrem que ele passa na
FEA logo no começo do trajeto (e provavelmente foi nesse ponto que vocês
desceram do ônibus quando chegaram). Se vocês pegarem o 8012-10
no ponto FAU I, também vão (em algum momento) chegar no metrô Butantã, mas
boatos dizem que os outros circulares vão mais rápido. Aí, basta ficar no
ônibus sentado (se conseguir um lugar!) até chegar no metrô Butantã, onde 
todos do ônibus vão descer.

\end{subsubsecao}

\begin{subsubsecao}{Como ir e vir do IME pelo P3}

{\bf Circulares durante dias de semana}

Para chegar no IME do P3, vocês tem duas opções: 8082-10 (terminal metrô 
Butantã) ou 8085-10 (circular interno P3 - CPTM). 

O 8082-10 faz o trajeto mais rápido (em média 7 minutos), porém ele demora para sair 
(em média 15 minutos), já o 8085-10 demora mais no trajeto.

O trajeto do 8082-10 é: Terminal P3, Vila Indiana, Biociências II e FAU I
(o ponto no qual vocês descem). Já o 8085-10 faz outro trajeto: Terminal P3,
Vila Indiana, ICB I e II, IPEN, HU II, Rio Pequeno, Prefeitura II, Poli e FAU II
(o ponto no qual vocês descem).

Para chegarem no P3 do IME, as opções são as mesmas, ou seja, o 8082-10, possuindo 
um trajeto menor e mais rápido, ou 8085-10. Se quiserem pegar o 8082-10 vocês 
devem andar até o ponto da Oceanografia, o trajeto leva em média 7 minutos, 
passando pelos pontos: Oceanográfico, Biociências I, ICB I e II e P3. 
Já para o 8085-10, andem até a FEA. Os pontos são: FEA, Poli, Prefeitura I, MAE, 
HU I, ICB III, Odontologia e P3.

{\bf Circulares durante finais de semana}

Durante o fim de semana, duas linhas de circulares trabalham: 8012-10 e 8022-10.
O 8012-10 faz o mesmo trajeto do 8082-10, já o 8022-10 faz um trajeto parecido com
o do 8085-10: Biomédicas II, Biomédicas I, IPEN, COPESP, Rio 
Pequeno, Pref II, Física e Oceanografia (o ponto no qual vocês descem).

Para chegarem no P3, se optarem por pegar o 8012-10, vocês andam até o ponto da Poli, e
seu trajeto é: Poli, Prefeitura I, MAE, HU I, Biomédicas III, Odontologia e P3. Já para 
o 8022-10, vocês andam até a Geociências: Geociências, Letras, História,
Farmácia, Biblioteca, Rua do Lago, Biomédias 1, IPEN, COPESP, HU, Biomédicas 3,
Odontologia e P3.

\end{subsubsecao}

\begin{subsubsecao}{Como ir e vir do IME pela Cidade Universitária}

Mesmo o trajeto mais comum usados pelos uspianos seja pelo metrô Butatã, sabemos 
que alguns de vocês, bixes, possuem a sorte de vir pela ViaCalamidade 
(vulgo linha 9 - Esmeralda). 
Assim, abaixo estão mais informações de como chegar no IME através da Portaria de 
Pedestres da CPTM.

{\bf Circulares durante dias de semana}

Nos dias de semana, apenas dois circulares passam no ponto da Portaria de Pedestres,
sendo eles o circular 1 (8085-10) e o circular 2 (8082-10). É aconselhável sempre 
pegar o circular 1 (8085-10), já que ele é EXTREMAMENTE rápido para chegar no IME 
(você também chegará com o circular 2 (8082-10), mas irá levar anos).
O circular 1 (8085-10) faz o seguinte trajeto para chegar no IME: Raia Olímpica, 
Praça do Relógio, Praça dos Bancos e Faculdade de Economia, Administração e Contabilidade
(FEA). É justamente no ponto da FEA que você deve descer.
O circular 2 (8082-10), como já dito, leva um tempo maior para chegar no IME, já
que ele segue o seguinte trajeto: Raia Olímpica, Psicologia 2, P2, Terminal USP, IPT,
Cocesp 1, Física, Oceanografia. Enfim, após percorrer todo esse caminho, você deve 
descer no ponto da Oceanografia. Acho que você já percebeu o porquê do circular 1
ser extremamente mais rápido!

Quando estiver saindo do IME e indo embora, os possíveis circulares para chegar até a
Portaria de Pedestres são, novamente, o circular 1 (8085-10) e o circular 2 (8082-10).
O circular 1 é possível pegar no ponto da FAU II e o circular 2 é possível pegar no ponto
da FAU I. Novamente, o circular 1 faz um caminho mais rápido para chegar até a portaria. 
Então, sempre que possível, opte por usar ele! CUIDADO para não pegar o circular 2 no 
ponto da Oceanografia, pois o 8082-10 que passa naquele ponto está indo em direção ao P3
(você até chega na portaria, mas vai rodar a USP toda).


{\bf Circulares durante finais de semana}

Se você tiver o azar de ir para a USP final de semana e precisar usar o circular, se prepare
para esperar horas e horas. Abaixo estão informações dos ônibus que você pode pegar
para ir e vir do IME.
Nos finais de semana, os circulares citados acima (8085-10 e 8082-10) não funcionam. 
Desse modo, apenas um circular passa na Portaria de Pedestres: o 8022-10. 
Para chegar no IME, você deve descer no ponto da FAU II.
Ao ir embora, novamente, você tem apenas uma opção de circular, sendo o 
8012-10. Você pode pegar ele no ponto da FAU I. 

\end{subsubsecao}

\begin{subsubsecao}{Linhas intermunicipais}

Agora, se vocês moram mais longe ainda (outra cidade, outro estado, outro país,
outra dimensão...) e não querem ou não podem se mudar para São Paulo, existem
algumas linhas de ônibus fretados para cidades mais próximas (ou não). Se a
cidade de vocês não estiver aí, procurem se informar a respeito, pois não
significa necessariamente que não haja ônibus da USP para lá. Aí estão elas:

\begin{itemize}
  \item {\bf Empresa Urubupungá}\\
    Tel: 0800 11 4777
    Site: {\tt www.urubupunga.com.br}\\
    280BI1- São Bernardo do Campo (Centro)\\
    Cor: Cinza\\
    Onde pegar para sair da USP: ponto FAU II\\
    Para mais informações sobre a rota e a tarifa dessa linha, acesse
    \url{http://itinerario.urubupunga.com.br:8080/itinerario/ItinerarioLinha.aspx?lin=46\&emp=1}

  \item {\bf Fretados Jundiaí - USP}\\
    Viação MIMO\\
    Tel: (11) 4606-8222\\
    {\tt www.viacaomimo.com.br}\\
    Principais horários:\\
    Ida: 6h00; 6h30; 7h00; 12h50; 18h00 (na Rodoviária de Jundiaí)\\
    Volta: 11h40; 15h00; 16h40; 17h05; 18h10; 21h00; 22h49 (No ponto FEA)

  \item {\bf ABC}\\
    Osmar: (11) 94710-8604\\
    Marcelo Antonio: (11) 94719-1783

  \item {\bf Santos}\\
    Arca Turismo\\
    (11) 5928-7961\\
    \url{http://www.arcaturismo.com.br}

\end{itemize}

\end{subsubsecao}

\end{subsecao}


\begin{subsecao}{Horário dos Portões: Veículos no Campus}

Saibam por onde entrar na USP. Lembrem-se de terem sempre a carteirinha USP, ou
o e-Card no celular, ou o comprovante de matrícula (nos primeiros dias), além de
RG, pois esses documentos podem ser solicitados principalmente nos horários de 
entrada controlada.
\begin{itemize}
  \item {\bf Portaria 1 (P1):} R. Afrânio Peixoto. Funciona 24h por dia todos os
  	dias, mas a entrada é controlada para pedestres e carros nos seguintes
	horários: todos os dias das 20h às 5h, aos sábados após as 14h, aos domingos
	e feriados o dia inteiro. É por onde entram os ônibus municipais.

  \item {\bf Portaria 2 (P2):} Av. Escola Politécnica. De segunda à sexta,
  	acesso livre das 5h às 20h e controlado das 20h às 0h. Aos sábados, acesso
	somente para pedestres: liberado das 5h às 14h e controlado das 14h às 0h.
	Fechada aos sábados (para veículos), domingos e feriados. Única entrada para
	caminhões.

  \item {\bf Portaria 3 (P3):} Av. Corifeu de Azevedo Marques. Tem o mesmo
  	horário de funcionamento do P1 para os pedestres. Para os carros, tem o
	mesmo horário do P1 exceto aos domingos, feriados e madrugadas (0h às 5h),
	nas quais ele fecha, ao contrário do P1.

  \item {\bf Portaria P' (Plinha):} R. Eng. Teixeira Soares. Funciona de segunda
  	à sexta das 6h às 18h. Fechado de sábados, domingos e feriados.

  \item {\bf Portarias de pedestre (Mercadinho, São Remo, HU, CPTM e
  	Vila Indiana):} Funcionam de segunda à sexta das 5h às 20h, e de sábado
	das 5h às 14h. Acesso controlado de segunda à sexta das 20h às 0h.
	Nas portarias do Mercadinho e Vila Indiana, pode entrar em quaisquer
	outros horários, incluindo domingos e feriados, mas o acesso é controlado.

\end{itemize}

As linhas de ônibus municipais que passam por dentro da USP não operam aos
finais de semana, com exceção dos circulares, que entram na Universidade a
qualquer hora, e da linha 702U (Term. Pq. D. Pedro II - Cid. Universitária), que
funciona todos os dias das 5h às 0h (aproximadamente!). Se vocês vêm de carro,
saibam que a universidade dispõe de bolsões de estacionamento gratuito em torno
das Unidades.

\end{subsecao}
\pagebreak
\begin{subsecao}{Pontos de táxi}
Existem alguns pontos de táxi espalhados pela Cidade Universitária. As frotas
operam de segunda à sexta-feira, das 7h às 23h, além de aos sábados, das 7h às
17h. Eis suas localidades:

\begin{itemize}
\item Praça das Agências Bancárias\\
Fone: (11) 3091-4488

\item Praça da Reitoria\\
Fone: (11) 3091-3556

\item Hospital Universitário\\
Fone: (11) 3091-3536
\end{itemize}
\end{subsecao}

\end{secao}
