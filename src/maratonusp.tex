\begin{subsecao}{MaratonUSP}

\figurapequenainlineapertada{maratonusp}

O MaratonUSP é o grupo de programação competitiva da USP que tem como foco o
preparo de estudantes para a Maratona de Programação e Olimpíadas de Informática.

No início do ano, o grupo promove o BixeCamp que são aulas direcionadas aos bixes
para o estudo de algoritmos e estruturas de dados, por isso, não deixe de
participar delas. Assim, o MaratonUSP é, atualmente, referência para o
Brasil inteiro por suas diversas conquistas em competições e pelo seu canal do
YouTube, com aulas assistidas por milhares estudantes de programação competitiva
pelo país.

O ambiente de cooperação faz com que o grupo tenha resultados incríveis na
Maratona, sendo Tetracampeão da fase brasileira e tendo participado 16
vezes da etapa mundial da competição, em países como a China, Tailândia,
Rússia, Estados Unidos e muito mais - tudo de graça! Vale lembrar que a própria
fase nacional acontece em um local diferente do país todo ano, o que também gera
uma ótima oportunidade para conhecer mais lugares do Brasil.

Por fim, a vida de Maratoneire não impacta só sua passagem pela universidade,
ela também abre portas para o futuro. É bastante comum que ex-participantes
trabalhem em empresas como Google, Meta e Microsoft, tanto no Brasil,
quanto no exterior. As habilidades desenvolvidas durante as competições,
viajando pelo mundo, dando aulas, preparando vídeos para o YouTube ou
simplesmente estudando com colegas durante uma tarde serão úteis tanto no
mercado de trabalho quanto na academia, além de render anos de muita diversão!

\begin{description}
\item[Facebook:] \url{https://facebook.com/MaratonUSP}
\item[Site:] \url{https://www.ime.usp.br/~maratona}
\item[Youtube:] \url{https://www.youtube.com/c/maratonusp}
% Atualizar o link do Telegram todo ano (esse é específico de 2024)
\item[Telegram:] \url{https://t.me/bixes2025}
\end{description}

\end{subsecao}
