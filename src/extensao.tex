\begin{subsecao}{Atividades de Extensão}

A extensão universitária é o pilar que conecta a universidade à sociedade. 
No IME, isso significa levar o conhecimento produzido nas salas de aula e 
nos grupos de pesquisa para fora dos muros da USP, além de trazer as 
demandas da comunidade externa para dentro do instituto.

Para vocês, a extensão é uma oportunidade incrível de aplicar o que aprendem 
na prática, desenvolver habilidades sociais e profissionais e, claro, 
conseguir as preciosas horas complementares (AACs). As atividades se 
dividem principalmente em:

\textbf{Grupos de Extensão:} São coletivos formados por estudantes e docentes 
com interesses comuns. O IME abriga grupos fortíssimos, especialmente na área 
de computação e tecnologia, como o FLUSP (focado em software livre), o Hardware 
Livre (focado em eletrônica e cultura maker), o USPCodeLab (inovação tecnológica 
e ciência de dados) e os grupos de treinamento para Maratonas de Programação.

\textbf{Programas e Projetos:} Envolvem ações voltadas para a melhoria do ensino 
básico e divulgação científica. Um grande destaque é a Matemateca, o acervo de 
matemática e estatística do IME que realiza exposições interativas. Há também 
programas de formação de professores, como o PAPMEM, e iniciativas de inclusão 
digital.

\textbf{Cursos:} O IME oferece cursos abertos à comunidade, organizados pelo 
CAEM (Centro de Aperfeiçoamento do Ensino de Matemática) ou através dos 
tradicionais Programas de Verão e Inverno. Vocês podem participar tanto como 
alunes quanto, futuramente, na monitoria e organização.

Fiquem de olho nos editais e nas redes sociais das atividades para se 
inscreverem ou se tornarem voluntáries!

\begin{description} 
  \item[Site Geral:] https://www.ime.usp.br/extensao/ 
  \item[Grupos:] https://www.ime.usp.br/extensao/grupos/
  \item[Programas:] https://www.ime.usp.br/extensao/programas/
  \item[Projetos:] https://www.ime.usp.br/extensao/projetos/
  \item[Cursos:] https://www.ime.usp.br/extensao/cursos/
\end{description}

\end{subsecao}
