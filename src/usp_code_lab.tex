\begin{subsecao}{USPCodeLab}

\figurapequenainlineapertada{uspcodelab}

O USPCodeLab é um grupo de extensão que tem por objetivo criar um espaço colaborativo 
para criar e incentivar o desenvolvimento de tecnologia na USP. Nosso foco é aprender 
na teoria e na prática ferramentas e técnicas de desenvolvimento de software que 
permitam solucionar problemas do mundo real.

Com a nossa iniciativa dev.learn(), oferecemos anualmente o curso Webdev sobre desenvolvimento web 
(HTML, CSS, JavaScript e frameworks). Também formamos grupos de estudos para desenvolver projetos 
de software propostos pelos próprios membros. Os projetos podem ser de nível básico ao avançado e 
alguns exemplos são um sistema de reserva de armários do CAMat e o CodeClass, a plataforma para 
distribuição dos cursos do CodeLab. 

O USPCodeLab também organiza hackatons (competições de programação em que os participantes elaboram
soluções sobre um tema proposto) como o HackFools e o Hackathon USP, o maior hackathon universitário
de São Paulo.

Sigam nosso perfil do Instagram (@uspcodelab), canal do Youtube (@CodeLabBr) e entrem no nosso grupo
do Telegram para participar da nossa comunidade! Participem do USPCodeLab!

\end{subsecao}
