\begin{subsecao}{Comissão de Inclusão e Pertencimento- CIP}

\textbf{O que é a CIP?}

A Comissão de Inclusão e Pertencimento do IME-USP é composta 
por pessoas de diferentes setores do IME e tem por objetivo promover e 
zelar por um ambiente que valorize e acolha a pluralidade através de atividades 
relacionadas a inclusão, pertencimento, diversidade e equidade no âmbito do IME.

\textbf{Quando procurar a CIP?}
\vspace{-15pt}
\begin{itemize}
  \item Solicitar adaptações pedagógicas ou funcionais para pessoas com TEA e/ou 
  outras condições neurodivergentes.
  \item Buscar ajuda ligada a questões de saúde mental, seja para você ou para 
  outra pessoa do IME.
  \item Buscar acolhimento ou denunciar comentários preconceituosos e atitudes 
  inadequadas nos espaços do IME.
  \item Procure a CIP mesmo em caso de dúvida. Se a sua demanda não for de nossa 
  competência, podemos ajudar a encaminhar para outras instâncias.

\textbf{Onde nos encontrar?}

Atendimento presencial:
Sala 260, Bloco A do IME-USP
(2º andar – em frente ao elevador)

Horário de atendimento:
Das 13:00 às 19:00, de segunda a sexta

\textbf{Conheça a gestão:} 

\textbf{Professoras}: 
\vspace{-15pt}
\begin{itemize}
  \item Cristina Brech ({\tt brech@ime.usp.br})
  \item Renata Wassermann ({\tt renata@ime.usp.br})
\end{itemize}

Para mais informações, envie um e-mail para {\tt cip@ime.usp.br} ou acesse:
\begin{itemize}
  \item Site: \url{https:www.ime.usp.br/cip}
\end{itemize}


\end{subsecao}
