\begin{secao}{Utilidades}

\begin{subsecao}{Na WEB}

\url{https://usp.br} - Página da USP. Aqui vocês encontrarão notícias e eventos da
universidade, bem como informações gerais.

\url{https://ime.usp.br} - Página do IME. Nesse link, vocês poderão ver detalhes sobre
os cursos e obter informações sobre a faculdade.

\url{https://uspdigital.usp.br/jupiterweb} - Sistema JúpiterWeb. Aqui vocês vão
encontrar a grade horária e, mais tarde, poderão fazer matrícula nas matérias
que irão cursar no semestre. Além disso, podem acompanhar o pagamento de
bolsas e auxílios por ele e também ter acesso ao seu histórico escolar com
as suas notas.

\url{https://portalservicos.usp.br} ou \url{https://olimpo.usp.br} - Portal com todos os
serviços digitais da USP. Vocês vão reparar que a aba ``Graduação'' na verdade
é o JúpiterWeb disfarçado, então na prática vocês podem só acessar o link acima
que dá na mesma (só é um pouco menos bonito porque é um site mais antigo).

\url{https://edisciplinas.usp.br} - É bixes\dots Acham que vai ser essa moleza pra
sempre? Se vocês acham, estão muito enganados! Daqui a pouco vocês vão receber
uma senha para poder enviar seus EPs (vide glossário) nesse endereço\dots
(e não adianta fazerem chantagem emocional que o e-Disciplinas só vai aceitar
até 23h59\dots Não entendeu? Vocês vão entender\dots)

\url{https://instagram.com/spottedimeusp/} - Viu alguém interessante? Quer mandar uma
cantada pro crush, mas tem vergonha? Manda um spotted! Afinal, só o $x$ deve
ficar isolado. Ou achou alguém bacana e com gostos parecidos, que tal mandar um
spotted para fazer novas amizades?

\url{https://www.instagram.com/usp_spotted/} - Spotted da USP. Serve para a mesma
coisa que o Spotted do IME, só que é para USP toda.

% \url{ www.xkcd.com} - Webcomic sobre matemática e computação. Origem de muitas
% piadas que vocês ouvirão por aí.

E por fim\dots

\url{https://google.com.br} - Tudo o que vocês precisam tá no google. Se não
estiver, então não existe. Aproveitem e procurem o Código de Ética da USP (ou
baixem ele em \url{http://www.prg.usp.br/wp-content/uploads/CodigoEtica.pdf})

\end{subsecao}

\begin{subsecao}{Apps}
	
Alguns dos apps disponíveis na AppStore e PlayStore oferecidos pela própria USP
que podem ajudar. Para ver a lista completa, procure pelo desenvolvedor
``Universidade de São Paulo'' em qualquer uma das duas lojas.

{\bf e-Card USP} - Esse é o app mais importante nesse começo de ano, enquanto
sua carteirinha não chega (ver seção Cartão USP no Júpiter). O aplicativo é uma
carteirinha virtual que te dá acesso a tudo que o cartão físico permite,
inclusive comer no bandejão.

{\bf Cardápio+ USP} - Contém o cardápio dos bandejões e mais outros serviços
da SAS, como creche, saúde mental e moradia. Lá vocês também poderão conferir 
se o bandejão está fechado ou funcionando com o horário reduzido.

{\bf Guia USP} - Mapa dos principais lugares da USP, incluindo os institutos,
restaurantes, bancos e mais. Mostra a localização dos circulares em tempo quase
real, mas não é muito preciso e às vezes trava.

{\bf Campus USP} - Permite acessar a central da Guarda Universitária
diretamente no caso de uma emergência. Contém um mapa com as ocorrências
recentes de segurança.

{\bf Disque Trote USP} - Use esse app para denunciar caso sofra um trote
violento. Se preferir conversar mesmo, também pode ligar para o número de
telefone que está mais embaixo.

\end{subsecao}

\begin{subsecao}{Telefones}

{\bf Disque Trote:} {\tt 0800-012-1090} --
\underline{Não deixe de ligar se você sofrer algum trote violento!!}

{\bf SAS:} Seção de Passe Escolar: {\tt (11) 2648-1863} ou {\tt (11) 2648-1865}

{\bf Serviço de Graduação do IME (mais conhecido como Seção de Alunos):} {\tt (11) 3091-6149}

{\bf Superintendência de Tecnologia da Informação (STI):} {\tt (11) 3091-6400}

%{\bf Plantão de Cálculo e Álgebra Linear (serviço gratuito):} {\tt 3037-1773}
%Diziam que era o orelhão do ime, vai que isso volta, resolvi só comentar^^

\end{subsecao}
\end{secao}
