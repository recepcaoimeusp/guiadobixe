\begin{subsecao}{Comissão de Acolhimento da Mulher - CAM}

A Comissão de Acolhimento da Mulher (CAM) foi criada em 2016 a partir de uma iniciativa do Coletivo Mulheres do IME. A proposta surgiu após quase um ano de reuniões abertas e debates amplos, envolvendo a comunidade, e foi apresentada à diretoria do instituto. Sua criação foi aprovada por unanimidade pela Congregação do IME, reforçando o compromisso institucional com o cuidado, o respeito e a igualdade.

Somos uma comissão assessora da diretoria do IME cuja principal função é oferecer acolhimento e escuta às mulheres que vivenciam situações de discriminação de gênero, assédio moral ou sexual e violência contra a mulher, sempre que essas situações envolverem pessoas da comunidade do IME ou ocorram em suas dependências. Nosso acolhimento é feito com respeito, sigilo e cuidado, e é direcionado tanto a mulheres cisgênero quanto a mulheres transgênero.

Sabemos que situações de discriminação e violência tornam a vivência universitária mais difícil e podem gerar sentimentos de medo, desrespeito e não pertencimento. A CAM existe para que nenhuma mulher do IME se sinta sozinha diante dessas situações. Nosso trabalho contribui para o enfrentamento institucional da violência contra a mulher, da desigualdade de gênero e dos impactos da cultura patriarcal no ambiente acadêmico.

Além do acolhimento em momentos difíceis, a CAM também promove encontros e atividades de convivência para fortalecer laços e criar um espaço seguro e acolhedor entre as mulheres do IME. Realizamos encontros informais, como piqueniques, jogos de baralho, rodas de conversa e atividades esportivas, onde podemos conversar, trocar experiências e nos conhecer para além da sala de aula. Mesmo com a baixa representatividade feminina nos cursos e nas pós-graduações das áreas de exatas, somos muitas: mulheres com histórias diversas, sonhos diferentes e desafios em comum. A CAM está aqui para acolher, apoiar e caminhar junto com você durante sua trajetória no IME.

Conheça a gestão: 

\textbf{Professoras}: 
\vspace{-15pt}
\begin{itemize}
  \item Estefani Moraes Moreira ({\tt estefani@ime.usp.br})
  \item Gisela Tunes ({\tt tunes@ime.usp.br})
\end{itemize}

\textbf{Funcionárias}: 
\vspace{-15pt}
\begin{itemize}
  \item Rosana Benedetti ({\tt rosanab@ime.usp.br})
  \item Stela Madruga ({\tt stela@ime.usp.br})
\end{itemize}

\textbf{Alunas - RD}: 
\vspace{-15pt}
\begin{itemize}
  \item Alice Hungria ({\tt alicehungria@usp.br})
  \item Cindy Hanna ({\tt cindyhanna@usp.br})
\end{itemize}

As mulheres que procurarem a comissão poderão, se quiserem, indicar com qual ou 
quais de seus membros desejam conversar.

Para mais informações, envie um e-mail para {\tt cam@ime.usp.br} ou acesse:
\begin{itemize}
  \item Site: \url{https://www.ime.usp.br/~cam}
  \item Instagram: \url{https://www.instagram.com/camimeusp/}
\end{itemize}


\end{subsecao}
