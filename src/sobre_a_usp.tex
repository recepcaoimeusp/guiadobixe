\begin{secao}{Um Pouco Sobre a USP}

Vocês, que estão conhecendo a USP pela primeira vez, devem saber desde cedo que aqui
há muita burocracia. É bom que se acostumem com ela, já que vocês terão que enfrentá-la.

O reitor é presidente do maior órgão da USP, o Conselho Universitário (abrevia-se
C.O. para evitar frases do tipo: ``Vou ter uma reunião no CU hoje'', ``O CU não
está funcionando muito bem esse semestre'', ``Os alunos não têm acesso ao CU''
etc.) que determina TODOS os rumos da universidade.

Abaixo dele vêm as coordenadorias e unidades. O SAS (Superintendência de Assistência Social),
por exemplo, é o departamento responsável pelos serviços
que a universidade oferece (não é a melhor coisa do mundo, mas oferece) para a
comunidade universitária: ônibus circulares, alimentação através dos restaurantes
universitários (famosos bandejões), moradia para estudantes (CRUSP) etc.

Alguns outros lugares que vocês devem saber que existem são o HU (Hospital
Universitário), o CEPE (Centro de Práticas Esportivas, leia ``cepê''), o banheiro
da FEA (Faculdade de Economia, Administração e Contabilidade), que fica na
frente do IME e é uma das faculdades mais bem abastecidas financeiramente
(apelidada ``carinhosamente'' de Shopping), a Física (certas aulas de laboratório são lá)
 e a querida Faculdade de Educação (para o pessoal da Licenciatura).

\begin{subsecao}{e-Disciplinas e E-mail USP}

Bixes, atenção e cuidado com o e-mail que vocês criaram (ou vão criar)! É um
email@usp.br, do sistema ``oficial'' de e-mails da USP, o que significa que é
através dele que a Universidade, o Jupiterweb, a diretoria do IME, o CAMat e
alguns desocupados lhes enviarão comunicados oficiais, o que pode envolver as
oportunidades diversas de intercâmbios, estágios, monitorias e todas as outras coisas
que vocês, simples bixes IMEanos, sejam capazes de se imaginar fazendo. Por este e-mail
vocês receberão informações antes mesmo das principais serem colocadas nos paineis eletrônicos que
ficam espalhados pelo IME (é, nem todas são colocadas nesses paineis não).
Com as aulas on-line ou não, o e-mail sempre foi uma das principais formas de
comunicação (às vezes a única) entre nós alunos e todos os outros grupos, pessoas,
órgãos etc. que compõem as comunidades IMEana e da USP, então é sempre bom verificar
a caixa de entrada do seu e-mail com frequência! 

Além disso, peçam para os veteranes falarem mais um pouco de algumas 
(das \textbf{muitas}) vantagens de se ter um e-mail USP e aproveitem tudo aquilo que 
vocês agora têm direito!

A USP tem, há alguns anos, um ambiente virtual de aprendizagem para dar apoio
às disciplinas: o Moodle da nossa universidade, oficialmente conhecido como
e-Disciplinas (é comum chamarmos dos dois jeitos e há alguns professores que
se referem a ele como ``Stoa'', que é um nome bem antigo!). Você pode acessá-lo em
\url{https://edisciplinas.usp.br/}, fazendo login com o e-mail USP e a senha única.

Na página inicial, você provavelmente encontrará links para as páginas da maioria
das disciplinas que vai cursar (se não todas). Muitos professores não utilizam o 
e-Disciplinas como apoio aos seus cursos, às vezes fazem isso nos seus próprios 
sites institucionais ou simplesmente não usam a internet para nada. 

\end{subsecao}

\begin{subsecao}{Grupos no Whatsapp}

Com o intuito de organizar as atividades referentes a uma disciplina e ser uma
maneira prática e efetiva de passar recados, costumamos utilizar grupos do WhatsApp
para cada uma das matérias. Caso os links ainda não tenham sido criados,
vale a pena juntar seus amigos para criar os grupos no começo do semestre e espalhar
para os outros alunos.

Dentre as principais funções do grupo do WhatsApp estão:

\begin{enumerate}
%\item Colocar o link para acessar a plataforma on-line das aulas na descrição pra facilitar pra todo mundo;
\item Compartilhar os materiais e listas disponibilizados pelo docente da disciplina;
\item Passar recados importantes, sejam datas para entrega de trabalho, provas etc.;
\item Tirar dúvidas sobre a matéria com os coleguinhas.
\end{enumerate}

Pra ficar por dentro de tudo o que acontece na matéria, não se esqueça de procurar o
link e fazer parte dos grupos de whats das matérias que você está cursando!! :)

\end{subsecao}

\begin{subsecao}{Eduroam}

O Eduroam (EDUcation ROAMing, não tem nada a ver com alguém chamado Eduardo) é um
serviço de wi-fi que \sout{às vezes} provê acesso à Internet dentro de diversas
universidades (inclusive a USP). Mesmo que você seja apenas aluno da USP, se viajar
para outra universidade, no Brasil ou no exterior, que possui o Eduroam, ainda poderá
ter acesso à Internet usando o mesmo login!

Para usar o Eduroam, basta se conectar à rede \texttt{eduroam}, colocando no campo
de login o seu número USP seguido por @usp.br. Por exemplo, se o seu número USP for
\texttt{123456789}, então deverá colocar no campo de login \texttt{123456789@usp.br}.
No campo da senha, deverá colocar a sua senha única (a mesma senha que vocês usam
para logar no Jupiterweb).

Tem várias outras informações sobre as configurações na hora de fazer login que
\sout{ninguém entende, mas se algum de vocês entender,} estão no site
\url{https://eduroam.usp.br/como-usar/}. Nesse site também tem alguns programas para Windows que 
configuram o Eduroam automaticamente. (Dica: se for baixar eles, utilize o Instalador Alternativo) %REFTIME
Também existem cartazes espalhados pelo IME com os detalhes de login para diferentes tipos de 
dispositivo, então deem uma olhada nas paredes.

\end{subsecao}

\begin{subsecao}{O SAS}

A Superintendência de Assistência Social (SAS), que fica próxima à Praça do
Relógio, é o órgão da USP responsável pelo bem estar financeiro dos alunos — não
é lá a melhor coisa do mundo, mas ajuda... É onde vocês podem conseguir seus
benefícios, tais como auxílios financeiros, moradia gratuita (CRUSP)
ou auxílio alimentação (conhecido como vale-bandex) e dentistas gratuitos.
A SAS também é responsável pelo Setor de Passes Escolares.

Se algum de vocês, bixes, estudou a vida inteira em escola pública, tem baixa renda,
ou está passando por algum tipo de vulnerabilidade social, não pense que está sozinho nessa,
muitos também passaram por isso e vocês conseguem encontrar apoio aqui. Permanência
estudantil é um direito e não deve ser motivo de vergonha, portanto busquem
auxílio, pois podemos ajudar. Seguem as diversas alternativas para todos 
que precisam;

\textbf{Moradia e auxílios financeiros}

Se vocês infelizmente não têm tanto acesso a meios de locomoção, ou dinheiro para
pagar transportes ou mesmo repúblicas, saibam que diversos auxílios podem ser
oferecidos para vocês para amenizar a sua situação:

% Com a carteirinha (e-card) que você recebeu na matrícula, você deverá entrar
% na página do SAS (\url{http://sites.usp.br/sas/}) e solicitar a inscrição
% para o processo de moradia e alojamento. Lá você terá um formulário, três
% trilhões de coisas para preencher e assim que te chamarem, deve apresentar os
% documentos (hum!) necessários para a assistente (você irá “ganhar” uma). Na
% improvável hipótese do site estar fora do ar, como todo ano acontece, você terá
% que ir ao SAS e pedir alojamento na USP, lá no Bloco G do CRUSP.

%REFTIME

As inscrições dos ingressantes para o Programa de Apoio à Permanência e Formação Estudantil (PAPFE)
deverão ser feitas dentro dos prazos estipulados no EDITAL PAPFE 2024, que pode 
ser encontrado na página do SAS (\url{https://prip.usp.br/wp-content/uploads/sites/1128/2023/12/EDITAL-PRIP-PAPFE-2024_v201223.pdf}).
No Portal de Serviços Computacionais da USP \url{https://portalservicos.usp.br/}, vocês entrarão 
com os dados da sua conta USP para se inscrever, responderão a questionários e anexarão (um bilhão de) 
documentos comprobatórios da situação socioeconômica declarada no ato de inscrição.

Vale ressaltar, bixes, que vocês tomem o cuidado de não fazer o mesmo que alguns
bixes de outros anos, que confundiram alojamento (vaga no CRUSP) com o auxílio moradia (financeiro). São dois
requerimentos distintos e é bom que vocês fiquem atentos na hora de se inscrever, verificando
qual(is) tipo(s) de auxílio(s) vocês precisam.

Os requerimentos conterão perguntas sobre a situação socioeconômica, bem como os
documentos que deverão ser trazidos para comprová-la. Perguntas como renda
salarial, quantas pessoas contribuem com ela, números de bens móveis e imóveis,
situação habitacional, tipo de escola em que estudaram, se trabalham e há quanto
tempo, quanto gastam para vir à USP, o tempo de ida etc. Ainda há um espaço
para descreverem alguma particularidade não exposta nas perguntas que, obviamente,
receberá um parecer técnico.

No ato da inscrição, você pode optar entre as ruas modalidades mencionadas: o auxílio permanência
no valor de 800 reais, ou a vaga no alojamento (CRUSP), acompanhada de um auxílio de 300 reais. Como 
o número de vagas no CRUSP é pequeno, naturalmente essa modalidade é mais concorrida que o auxílio financeiro integral.
Caso você seja selecionado em qualquer uma dessas modalidades, recebe também o auxílio alimentação, que se trata de créditos
para carregar o cartão para o bandejão gratuitamente, todo mês, através do JúpiterWeb.

%\textbf{Bolsa-trabalho}

%Destina-se a alunos de graduação vinculados a projetos de extensão de serviços à
%coletividade. Os projetos são selecionados anualmente, de acordo com sua relevância
%para as finalidades da universidade pública e os estudantes vinculam-se por
%afinidade acadêmica ou científica. Cada bolsa é de 1 (um) salário mínimo por
%40 horas de trabalho mensais. Além da seleção socioeconômica feita pela
%DPS (Divisão de Promoção Social), há uma seleção técnica feita pelos supervisores
%dos projetos.

%Mas fiquem espertos! Para tudo tem prazo e o SAS não é obrigado a ficar esperando a
%boa vontade de aparecerem por lá de ninguém. Qualquer dúvida, liguem pro SAS.

\textbf{Atendimento odontológico gratuito}

Antigamente para vocês agendarem o atendimento odontológico gratuito era necessário fazer uma
carteirinha no HU (Hospital Universitário), em seguida comparecer ao Bloco G
do CRUSP com a carteirinha e agendar, porém hoje em dia, graças à tecnologia,
isso não é mais necessário: você só vai precisar mandar uns e-mails, preencher uns
formulários e esperar \sout{uma eternidade}.

Primeiro você precisa mandar um e-mail solicitando triagem para {\tt triagemodonto.sas@usp.br},
vão te responder com um formulário que você vai preencher e enviar de volta, após isso eles
te colocarão na fila de espera pra triagem. No final, deve demorar muitos meses pra
eles te chamarem e quando fizerem isso você só vai ter que comparecer ao Bloco G do CRUSP
munido do seu cartão USP.

No final será demorado, porém o serviço é gratuito e realmente ajuda. Eles também atendem casos
de emergência como por exemplo dor de dente, algum dente quebrado ou qualquer outra situação
do gênero. No caso de dúvidas, são poucos os que podem te ajudar nesse assunto, então o melhor
a se fazer é ligar pra lá: (11) 3091-3393.

%FIXME
\textbf{Setor de passe escolar}

Será nesse lugar que você resolverá boa parte dos seus problemas com o SAS e com o
Bilhete Único. Você pode se dirigir ao Bloco “E” do CRUSP para receber orientação sobre
as mais diversas burocracias em que está se metendo.

Para as linhas da EMTU, SPTRANS, METRO, vocês devem entrar no site do SAS e fazer
o pré-cadastro para o respectivo cartão (\url {http://sites.usp.br/sas/}) para que a USP envie os 
seus dados para as empresas de transporte. Talvez demore um pouco, pois a USP tem que
avisar para a SPTrans que vocês passaram na Fuvest. Fiquem atentos! Qualquer dúvida,
liguem para a seção de passe escolar do SAS: (11) 3091-3581. Vocês também podem entrar nos sites 
das empresas de transporte e procurar por mais informações lá.

Há também um cartão especial para alunos de universidades públicas provenientes
de outras cidades do interior de São Paulo. Se algum de vocês vem de algum desses domos
ignotos (como Resende, Caçapava ou Guaíra), pode se dirigir ao guichê da sua empresa de
transporte intermunicipal (Cometa, Danúbio Azul etc.), apresentar seu Cartão USP,
preencher um formulário e eles aguardarão confirmação do seu instituto. Então
você pega seu cartão na Seção de Alunos e, na compra de passagens entre São Paulo
e sua cidade-natal, paga 50\% do valor normal. Só não deixe isto por último na
sua lista de necessidades porque existe um período do ano em que os guichês
liberam seus formulários; em resumo, espiche suas orelhas e corra para a rodoviária.

\end{subsecao}

\begin{subsecao}{Atendimento Psicológico}

Existem inúmeros serviços na USP que oferecem atendimento psicológico
aos alunos, cada um com um foco e um público-alvo. Dentre eles, destaca-se
o Serviço ECOS, um serviço focado em três etapas: Escuta, Acolhimento e Ação.

O ECOS conta com um espaço de atendimento situado no campus Butantã, realizando 
escutas pontuais presenciais e sem necessidade de agendamento para reflexão e possíveis 
encaminhamentos às redes de cuidado interna ou externa à USP. As escutas são realizadas 
pela equipe ECOS, composta por profissionais de saúde e alunes bolsistas de diversas áreas 
do conhecimento, que desenvolvem projetos de iniciação científica. O programa não atende
emergências. Para situações de crise aguda em saúde mental, deve-se buscar um Pronto Socorro 
Psiquiátrico (a referência para o campus Butantã é o Pronto Socorro Lapa, Av. Queiroz Filho, 
313; telefone: 4878-1701).

O Programa articula continuamente uma rede de cuidados em saúde mental baseada nos
territórios dos campi USP, mobilizando ações e serviços universitários, do SUS, de 
outras políticas públicas e da sociedade, que visa garantir acesso a estes recursos 
pela comunidade universitária e fornecer um Mapa da Saúde Mental da USP.

A equipe do ECOS oferece apoio institucional a diferentes grupos, unidades e campi da 
USP, seja para a criação de ações e serviços relacionados à saúde mental, seja para a 
qualificação de iniciativas já existentes. Dessa forma, realiza o acolhimento a demandas 
coletivas, buscando potencializar ações locais e a participação das respectivas comunidades 
na promoção da saúde mental em cada território.

Caso você esteja precisando, o local é próximo ao bandejão Central, confira o endereço, contato e informações: 
Endereço: Rua do Anfiteatro, 181 – Favo 22 – Cidade Universitária – São Paulo/SP (em frente à portaria do Bloco C do CRUSP)
Horário de funcionamento: Segunda a sexta-feira, das 08h às 17h
Telefone: (11) 3091-8345
E-mail: ecos.prip@usp.br

Uma outra opção são os serviços oferecidos pelo Instituto de Psicologia (IPUSP).
Os principais serviços do IP são o LEFE, o Atendimento Presencial e o PAPO:

O LEFE se trata de um plantão psicológico oferecido à comunidade, em que
você pode conversar com um profissional num momento de crise. O plantão 
funciona na terça-feira às 17 h, no IP, e para ser atendido basta chegar 
entre as 16 e 17 h.

Mais informações sobre o LEFE: 
\url{https://www.ip.usp.br/site/plantao-psicologico-lefe/}

O Atendiento Presencial no IP é o serviço em que você pode ter acesso a
atendimento individual de forma contínua por um período mais longo. Essa
modalidade também exige inscrição, e os inscritos ficam em uma lista para
passar por triagem. O período com maior chance de ser selecionado é no inicio
do semestre, mas também é possível conseguir vagas em outros momentos em 
situações excepcionais. 

Mais informações sobre o Atendimento:
\url{https://www.ip.usp.br/site/atendimento-4/}

O PAPO é o Projeto de Apoio Psicológico Online, que foi criado durante a 
pandemia para oferecer de forma segura uma via de atendimento psicológico
para a comunidade. Para essa modalidade de atendimento é preciso se inscrever
previamente, no período indicado no site.
\url{https://www.ip.usp.br/site/covid-19-apoio-psicologico-online/}

O IP também oferece outros serviços, como o Ateliê Aberto e o Serviço
de Orientação Profissional. Você pode conhecer eles aqui:
\url{https://www.ip.usp.br/site/centro-escola-do-instituto-de-psicologia-da-usp-ceip/}


Bixes, sempre que precisarem, não tenham receio de procurar apoio
psicológico, por mais banal que vocês possam achar o motivo. Vocês
podem frequentar os serviços oferecidos pela USP para conversar sobre
qualquer coisa que esteja lhes incomodando, sejam professores,
matérias, EPs ou algo do tipo sem criar qualquer
problema para vocês no IME.

\end{subsecao}

% Utilize para começar uma nova página do lado esquerdo do Guia!

\begin{subsecao}{Bandejão}
%TODO Fazer as duas páginas ficarem na mesma "visualização" no guia impresso
%Ou seja, usar esquema Odd/Even do LaTeX.
% Isso foi deixado de lado nessa versão de 2016 para que o mapa dos circulares
% ficasse no lugar certo e não tivesse nenhuma página em branco
% (como o cleardoublepage ali em cima)
% 2017: não sei se entendi o que era para ser feito, mas usei o
%       \cleardoublepage para a imagem do calvin não ficar zuada.


Os bandejões, vulgarmente conhecidos como Restaurantes do SAS, são os lugares
em que vocês podem se alimentar razoavelmente a um preço analogamente razoável.
Os tickets custam R\$2,00, sendo estes carregados na carteirinha USP. Antigamente, o
lugar para carregar o ``bandejão único'' era no CARE (Centro USP de Acolhimento e 
Referência para Estudantes), perto do SAS e do bandejão Central, porém, em 2024, essa
opção deixou de funcionar. Além disso, é possível verificar o saldo atual e comprar 
mais créditos via Pix no site {\tt https://uspdigital.usp.br/rucard/} ou pelo
aplicativo ``Cardápio+ USP'', disponível para Android e iOS (não se esqueça de olhar 
também os outros aplicativos da USP disponíveis, pois podem ser muito úteis).
Segundo os meios oficiais (e-mails enviados pela usp), a opção de boleto não 
estaria mais funcionando, porém, alunos relataram que essa opção ainda aparece 
para eles. Portanto, caso queiram e consigam recarregar o RUcard via boleto, fique 
sabendo que os créditos demoram até 03 (três) dias úteis para estarem disponíveis.
Se realizarem o pagamento pelo Pix, os créditos podem ser usados dentro de alguns
minutos.
Dica: Confira o saldo da carteirinha no site ou no aplicativo antes de tentar usar!

Existe uma lenda que sobremesas especiais aparecem em certos bandejões perto de
datas comemorativas. Mas não se engane - normalmente, elas não aparecem no
cardápio (talvez para não ficar muito cheio de gente).

\figuragrande{bandex_calvin}

O cardápio semanal do bandejão pode ser visto no site {\tt
http://sites.usp.br/sas/} ou pelo aplicativo ``Cardápio+ USP'', mas se vocês
estiverem com preguiça de ver em um desses lugares, é bem provável que um veterane 
já saiba e resolva informar, se questionado com muita educação.

O cardápio é geralmente composto de arroz (com opção normal e integral), feijão,
prato principal (carne/ovos), acompanhamento (legumes ou verduras refogadas,
cremes, molhos), salada (algumas folhas), sobremesa (frutas e/ou, às vezes, algo mais
requintado), pãozinho (normal e integral) e suco (de diversos sabores: azul, vermelho,
amarelo, etc.), além de temperos genéricos (jamais perguntem do que eles são feitos). 
Se vocês são vegetarianos, o acompanhamento nunca contém carne (e estão tentando 
fazer com que sejam veganos também), e todos os bandejões têm uma alternativa 
vegetariana de PVT para o prato principal, normalmente vindo na forma de ração,
podendo também aparecer nas formas de escondidinho, lasanha, kibe e outros
(vale a pena experimentar o escondidinho da Central).


Não se esqueçam de levar as suas canecas do Kit-bixe se forem comer nos
bandejões, tanto para ostentar a posição de bixes do IME, quanto para salvar o
meio ambiente, evitando assim o uso de copos descartáveis. Os bandejões não 
disponibilizam copos e não aceitam o uso de garrafas nas máquinas de suco, por 
isso, é muito importante que vocês sempre tenham um copo ou caneca para conseguir 
tomar o líquido colorido saborizado do bandeco.


PS: O Efeito Bandex é proporcional à quantidade de salitre utilizado em cada
bandejão.\\
PS 2: Nunca, em hipótese alguma, jamais, visite a cozinha do seu bandejão de
preferência, pois você corre o risco de nunca mais almoçar na vida. Como já
dizia o velho sábio ``A ignorância é uma virtude''.\\
PS 3: Playstation 3.

Consultem a tabela abaixo para decidir em qual dos bandejões vocês vão comer.

\begin{figure}[!htbp]
\begin{center}
 	\includegraphics{img/tabela_bandeco_2023.pdf}
\end{center}
\end{figure}


O horário de funcionamento dos bandejões é o seguinte:\\
\textbf{Café da manhã:} 7h às 8h30 (somente no Central, de segunda à sábado)\\
\textbf{Almoço:} 11h15 às 14h15 (em todos os bandejões de segunda à sexta, de sábado 
somente do Central e de domingo somente na Química)\\
\textbf{Jantar:} 17h30 às 19h45 (somente na Química, na Física e no Central de segunda à sexta)\\

\textbf{IMPORTANTE:} o funcionamento dos bandejões pode ser alterado ao decorrer do ano,
principalmente nos feriados, por isso recomendamos que vocês fiquem atentos ao 
aplicativo e às placas informativas presentes nos bandejões.

Para mais informações, visitem o site do SAS: \url{http://sites.usp.br/sas/} ou
utilizem o aplicativo ``Cardápio+ USP''.

\end{subsecao}

\begin{subsecao}{Outros lugares para comer na USP}

Caso vocês não queiram comer no bandejão, seja porque estão com medo do
prato do dia ou simplesmente porque querem comer algo diferente (e
possivelmente de sabor melhor), saibam que há alguns outros lugares na USP onde
vocês podem comer; por exemplo, os que estão na seguinte lista.

%REFTIME
\textbf{Importante:} as informações referentes a valores não estão atualizadas com
os preços vigentes em 2024. Interpretem onde estiver escrito ``custa R\$ $x$'' 
como ``custa R\$ $x$ ou um tanto a mais''.
% Acredito que nenhum dos preços esteja atualizado, então generalizei

\textbf{Lanchonete da Física}

Fica um pouco antes do bandejão da Física, nela vocês podem almoçar pagando R\$
63,00 o quilo, sempre tem uma razoável quantidade de saladas e pratos
quentes, além disso, tem churrasco com uma boa variedade de opções.
Além do almoço por quilo, são vendidos alguns salgados e lanches como sanduíches e
beirutes. E logo em frente à lanchonete, há um lugar que vende cookies por R\$
8,50 (3 unidades). Vocês vão sentir o cheiro MARAVILHOSO dos cookies quando forem bandejar.

\textbf{Restaurante do IPEN}

Fica na parte de cima da rua entre o IF e o Parque Esporte Para Todos, para
entrar lá vocês precisam da carteirinha USP. O quilo custa R\$49,90 para sócios da marinha 
e R\$54,99 para não sócios. Infelizmente não há uma variedade muito grande de comidas lá e o
restaurante só abre para pessoas de fora às 13h.

\textbf{Lanchonete da FAU}

Localizada dentro da FAU assim que você sobe a rampa. Lá além de lanches é
servido almoço tanto por quilo (custa R\$43,90) como prato feito. Tirando o por
quilo, qualquer outra coisa deve ser paga no caixa antes de retirar. % Desde o ínicio da pandemia, 
%a Lanchonete da FAU se encontra fechada, mas pode ser que em breve ela volte a funcionar.
% Acredito que voltou, mas não tenho certeza se ainda tem o almoço por quilo. 
Além disso, no terceiro andar da FAU tem a mesinha de doces que os alunos deixam lá
para vender. A variedade e quantidade de doces varia bastante com o dia e o
preço deles costuma ser no máximo R\$2,50.

\textbf{Restaurante da FEA}

Fica atrás da FEA, lá o quilo tem bastante coisa diferente e algumas até que
sofisticadas, porém o preço do quilo lá é R\$85,90, provavelmente é o mais caro
da USP. Geralmente é frequentado por professores e funcionários da USP. Além do
self-service, há algumas opções de prato feito que custam R\$36,00.

\textbf{Trailers de lanche da \sout{ECA} Química}

Ficam na calçada com a Avenida Lineu Prestes, em frente ao Instituto de Química. 
Nesses carrinhos vende pastel, sanduíches, churros, tapioca, salgados, açaís, yakissobas e, 
inclusive, pratos feitos. Os preços dos lanches variam entre R\$11,00 e R\$22,00 e os das tapiocas 
entre R\$5,00 e R\$17,00. As opções de sanduíches e a quantidade de sabores de tapioca 
são variados e todos em geral são muito bons. Para os pasteis, vocês podem escolher até 5 recheios, 
e o preço varia de acordo com a quantidade de ingredientes escolhidos. O prato feito costuma custar 
entre R\$18,00 e R\$25,00 e o yakissoba entre R\$25,00 e R\$35,00.

\textbf{Cachorro-quente da Reitoria}

Fica na rotatória da biblioteca Brasiliana (ou a rotatória depois do CEPE, se assim preferir),
lá o cachorro-quente pode ser feito tanto no pão de cachorro-quente como na baguete. Além disso,
há a opção de pôr catupiry e de pôr uma salsicha adicional. O preço do cachorro-quente varia
entre R\$12,00 e R\$16,00.

\textbf{Poke da Reitoria}

Tem um lugar que vende Poke (um prato havaiano que mistura peixe cru em cubos,
arroz, shoyu e frutas) ao lado do cachorro-quente. Esta iguaria custa
aproximadamente R\$23,00 e você pode escolher os ingredientes que vão no seu prato,
tipo Spoleto. Dá pra fazer um Poke vegano, com cogumelos em vez de peixe. Além de
Poke, também vende Temakis, Hot Rolls, e até Açaí. A qualidade é boa, mas as filas
podem ser demoradas.

\textbf{Pastel do IEE}

Esse lugar maravilhoso fica a apenas alguns passos do IME. Apenas às terças-feiras,
em um canto escondido do IEE (pergunte a um veterane como chegar!) fica o famoso pastel do IEE.
Há muitas opções de sabores, além de poder comprar um caldo de cana para acompanhar (também tem 
várias opções de sabores!).

% \textbf{Restaurante do Viveiro}

% Esse restaurante foi inaugurado no final de 2023, e também fica muito perto do IME, próximo ao ponto 
% de ônibus ao lado do IO. Fica aberto das 11h às 14h30, e serve [...]
% Fica aqui um começo de texto se alguém quiser adicionar no futuro. 

%FIXME GAMBIARRA
\pagebreak
\end{subsecao}

\end{secao}
