\begin{subsecao}{Truco}

O Truco é um jogo de boteco, e você já deve ter jogado ou visto alguém jogar em
algum momento da sua vida (não era só você que passava o intervalo da escola ou
do cursinho, e até algumas aulas, jogando Truco...) Ele é jogado por 4
jogadores formando 2 duplas ou 6 jogadores formando 2 trios, que se sentam
alternados à mesa.

Utiliza-se um baralho sem as cartas 8, 9 e 10. No truco a ordem crescente das cartas é: 

\begin{center}
\textbf{4 5 6 7 Q J K A 2 3}
\end{center}

Ou seja, 4 é a mais fraca e 3 a mais forte (por enquanto...).

O jogo é divido em rodadas com 3 turnos cada. As duplas ganham um ponto (ou tento) se ganharem uma rodada, ganha o jogo  a dupla que alcançar 12 pontos primeiro.

\textit{Um turno:}

O carteador (também chamado de 'pé') embaralha o maço e dá ao jogador da
esquerda para que este corte o baralho (por cortar, entende-se dividir o baralho
em duas porções, para que apenas uma delas seja utilizada na distribuição das
cartas). 

Daí, distribui 3 cartas para cada
jogador e vira uma carta sobre a mesa. Essa carta determina qual será a \textbf{manilha}
do turno - ou seja, a carta mais forte do turno. A manilha será sempre a \textbf{carta
seguinte}, em ordem de tamanho, da virada. Por exemplo, se a carta virada for um J,
a manilha será o K e a ordem crescente das cartas vira: 4 5 6 7 Q J A 2 3 K. 

É importante ressaltar que,
entre as manilhas (carta mais forte do turno), existe uma hierarquia de naipe. A ordem crescente é:  

\begin{center}
Ouros ($\diamondsuit$) $\rightarrow$ Espadas ($\spadesuit$) 
$\rightarrow$ Copas ($\heartsuit$) $\rightarrow$ Paus ($\clubsuit$)
\end{center}

Isto é, Paus a mais forte e Ouros a mais fraca. Esta hierarquia é utilizada em caso de empate, ou seja, quando dois jogadores jogam manilhas de naipes diferentes.

A pessoa à direita do carteador (também chamada de 'mão') será a primeira a
jogar uma carta. O jogo roda em sentido anti-horário. Todos os participantes
deverão jogar uma carta na mesa seguindo a ordem de jogadores. Aquele que jogar
a carta mais forte ganha o turno e torna a jogar no próximo turno. Ganha um ponto 
a dupla ou trio que fizer dois dos três turnos, ganhando a rodada.

\textit{O truco:}

Na sua vez de jogar, um jogador pode pedir ``TRUCO!!'', aumentando o valor da rodada para 3 pontos. A dupla adversária pode fugir (com a dupla que pediu truco ganhando 1 ponto), jogar valendo 3 pontos, ou pedir ``SEIS MARRECO!'', aumentando mais ainda
o valor da rodada. O valor da rodada pode ser aumentado gradualmente para ``Nove''
ou ``Doze'', sempre oferecendo a oportunidade para a equipe adversária fugir,
com a dupla que pediu o aumento ganhando o valor atual da jogada (por exemplo, 
ganhando seis pontos ao fugir de um pedido de ``Nove'') e partindo para a próxima rodada. 

\textit{O turno melado:}

Quando o primeiro turno empata, por exemplo, com dois Ases
jogados por duplas diferentes e nenhuma manilha jogada,
o turno é dito 'melado' e o segunda turno
decide o jogo. Se no segundo turno ocorrer mais um empate, é o terceiro que decide
o jogo. Por fim, se ocorrer mais um empate no terceiro turno, nenhuma equipe leva
o ponto. Por outro lado, caso o empate ocorra apenas no segundo turno, vence a equipe que tiver vencido o primeiro turno. 

É importante destacar
que, se as duas cartas que empataram a rodada forem manilhas, neste caso em
específico, existe um desempate, que se dá pela força dos naipes. 

\textit{A mão de onze:}

Quando uma das equipes está com 11 pontos, cada jogador dessa
equipe pode checar as cartas do seu parceiro antes de decidir se joga ou não.
No caso de aceitarem o jogo, a rodada vale imediatamente 3 pontos (e não pode
ser trucada, sob pena de perder o jogo). No caso de não aceitarem, a equipe
adversária ganha apenas um ponto. 

\textit{A mão de ferro:}

Quando as duas equipes estiverem com 11 pontos, é chamada
mão de ferro, e será a última mão da partida: aquela que vencê-la
vence o jogo. Esta mão é jogada no escuro, ou seja, nenhum jogador
pode ver as cartas que receber, deve deixá-las na mesa viradas
pra baixo até o momento que jogar a carta. 

\end{subsecao}