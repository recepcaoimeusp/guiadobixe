\begin{subsubsecao}{Apoio BCC}

Desde 2011, o BCC conta com um projeto de melhoria do curso: o Apoio BCC.

Muites bixes, quando entram, não têm ideia do que é o curso de fato. Também não
sabem o que a USP pode lhes trazer de interessante.

Para tentar dar uma luz à bixarada, o pessoal do Apoio montou um Canal do YouTube:
\url{http://youtube.com/c/CienciadacomputacaoIMEUSP}, contendo vídeos sobre as 
disciplinas, entrevistas com ex-alunes, apresentações dos grupos de extensão, 
palestras e muito mais. A ideia aqui é apresentar o curso e suas possibilidades 
pelo ponto de vista de alunes, e mostrar tudo de legal que fazemos no BCC.

Também guardamos uma cópia de todos os TCCs a partir de 2000 para que a 
comunidade possa consultar e eventualmente se inspirar. Como o curso tem TCC, é 
legal tomar contato com esse tipo de material o quanto antes!

Se vocês quiserem se informar melhor sobre determinadas coisas do curso, como as
citadas acima, visitar o \textit{site} do Apoio é obrigatório! Site do Apoio: 
\url{http://bcc.ime.usp.br}

O Apoio BCC é a principal voz do BCC para que professores e alunes conversem e 
possam melhorar o curso, por isso fique atente ao seu \textit{e-mail USP} para 
anúncios do CAFÉ DO BCC, evento que promove esse encontro entre discentes e 
docentes, além de pesquisas sobre questões importantes do curso. Caso deseje 
participar do projeto, as inscrições são feitas entre o final do 2º e início 
do 1º semestre; portanto, NÃO DEIXE DE ACOMPANHAR SEU \textit{E-MAIL USP!}

\end{subsubsecao}